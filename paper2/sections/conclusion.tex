% Paper 2 - Conclusion
% Camera-ready draft from Kelly

This work extends the CHSH landscape introduced in Paper~1 by resolving its
temporal structure across coupling and noise. By measuring lag-resolved
correlations, curvature, echo strength, and susceptibility across a
high-resolution $(K,\sigma)$ grid, we find that CHSH memory in classical
two--oscillator dynamics is organized into three robust regimes: a
noise-dominated basin of trivial memory, a structured intermediate band with
mid-range enhancement, and a narrow ridge of strong instantaneous coherence but
vanishing temporal persistence.

The discovery of a universal memory threshold
$\sigma_{\mathrm{mem}} \approx 0.002$, far below the CHSH collapse curve
$\sigma_{\mathrm{c}}(K)$, highlights a sharp separation of scales between
instantaneous and lagged correlations. This separation imposes a concrete
limitation on the temporal coherence of CHSH-like dynamics: even systems that
exhibit strong instantaneous geometry cannot retain long-lag memory under
noise. The curvature and susceptibility analyses further show that the
intermediate regime contains nontrivial temporal structure that cannot be
inferred from the CHSH functional alone.

These findings suggest a new perspective for interpreting CHSH-like dynamics:
rather than a single coherent structure, the CHSH ridge is embedded within a
multi-layered temporal environment whose properties depend sensitively on both
lag and noise. This layered organization provides a principled vocabulary for
describing ``instantaneous coherence'' versus ``temporal coherence'' in
classical systems and sets the stage for extending the framework to larger
networks.

Paper~3 applies this temporal vocabulary to three-oscillator motifs---chain,
star, and triangle topologies---revealing how CHSH-like memory structures behave
when embedded in small networks with frustration, mediated coupling, and
cyclic flows. Together, the two papers establish a coherent path from
two-oscillator CHSH geometry to the topographic analysis of observer fields in
larger classical systems.
