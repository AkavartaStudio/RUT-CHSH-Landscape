% Paper 2 - Results Section 3.2: Memory Curvature Surface
% Camera-ready draft from Kelly

\subsection{Memory curvature across the CHSH landscape}
\label{sec:memory_curvature}

The analysis above identifies a sharp separation of scales between the noise
level required to destroy long-lag CHSH memory and the much larger scale
required to collapse the CHSH functional itself. To resolve the internal
structure of this ``memory layer'' we now examine how the temporal profile of
$\rho_S(\tau)$ bends as a function of lag and location in the $(K,\sigma)$
plane.

For each parameter pair $(K,\sigma)$ we evaluate the lagged CHSH
autocorrelation $\rho_S(\tau)$ on a small grid of lags
$\tau \in \{\tau_1,\tau_2,\tau_3,\tau_4\}$ and construct a discrete curvature
diagnostic $C_{\mathrm{mem}}(\tau_{\mathrm{mid}}; K,\sigma)$ at three midpoints
$\tau_{\mathrm{mid}} \in \{17.5, 37.5, 75.0\}$ using a second-order finite
difference across neighboring lags. Positive $C_{\mathrm{mem}}$ indicates that
the mid-lag echo lies above the linear interpolation between shorter and longer
lags (locally ``bulging up''), while negative $C_{\mathrm{mem}}$ indicates a
concave profile in which the mid-lag correlation is suppressed relative to a
linear decay.

Figure~\ref{fig:fig2_memory_curvature} summarizes the curvature statistics
across 1995 simulations (19 couplings, 21 noise levels, 5 seeds per point). At
short lag ($\tau_{\mathrm{mid}} = 17.5$), the curvature is predominantly
negative, with $C_{\mathrm{mem}}$ ranging from $-0.031$ to $+0.001$ and a
strong imbalance towards decay-dominated profiles (81 positive, 318 negative
cases). Very close to the present, CHSH memory therefore behaves as a
conventionally relaxing signal: once noise is introduced, the correlation
typically begins bending downward immediately.

At intermediate lag ($\tau_{\mathrm{mid}} = 37.5$), the sign distribution
reverses. Here $C_{\mathrm{mem}}$ remains modest in magnitude
($[-0.005, +0.007]$) but is overwhelmingly positive in sign
(339 positive, 60 negative). Across most of the high-$|S|$ region the mid-lag
echo sits \emph{above} the straight-line interpolation between shorter and
longer lags: as a function of $\tau$, the correlation profile develops a
shallow hump rather than a simple exponential or power-law decay. In other
words, there is a robust ``mid-range'' band in which CHSH memory is slightly
enhanced relative to the naive expectation from its short- and long-lag
behavior.

At long lag ($\tau_{\mathrm{mid}} = 75.0$), the sign distribution flips back
again: $C_{\mathrm{mem}}$ lies in $[-0.001, +0.005]$ with negative values now
dominant (43 positive, 356 negative). Far out in time, the system behaves as a
genuinely decaying process: whatever mid-range enhancement exists has been
exhausted, and the residual correlations drift back toward zero or slightly
negative values.

Taken together, these curvature results show that CHSH memory is not a
featureless exponential tail. Instead, the $(K,\sigma)$ plane contains a broad
intermediate band in which mid-range lags display a slight echo ``bulge,''
meaning that correlations are transiently stronger than suggested by both
earlier and later lags. This intermediate region is embedded between short-
and long-lag regimes in which curvature is dominantly negative. The resulting
three-zone structure is invisible to instantaneous diagnostics such as $|S|$
itself, and only becomes apparent once we track how the correlation profile
bends as a function of lag.

In addition to the broad sign structure, the curvature surface
$C_{\mathrm{mem}}(K,\sigma)$ exhibits small, isolated pockets of negative
curvature embedded within regions of positive curvature. These ``memory
hollows'' occur near the boundary between the coherent ridge and the
intermediate band: short- and long-lag correlations remain nontrivial, but the
mid-lag statistic dips below its neighbors, producing a local inversion of the
echo profile. This indicates that, in these narrow parameter zones, temporal
coherence briefly weakens at intermediate lag before recovering, revealing
fine-grained structure in the way CHSH memory is redistributed as noise is
introduced.

Curvature reveals how memory bends locally in $\tau$, but does not show how
echo behaves globally across the entire $(K,\sigma)$ field. To map this full
geometry, we next examine the echo surface $\rho_S(\tau=50)$, which exposes
the global memory plateau and its abrupt destruction.

The mid-lag enhancement occurs near $\tau \approx O(1/\lambda)$, consistent with
a transient coherence resonance where weak noise reinforces correlations before
eventual long-lag decay.


%=============================================================================
% FIGURE
%=============================================================================

\begin{figure}[H]
\centering
\includegraphics[width=0.9\linewidth]{figs/fig2_memory_curvature}
\caption{
    Memory curvature surface $C_{\mathrm{mem}}(K,\sigma)$ at mid-lag
    ($\tau_{\mathrm{mid}}=37.5$).
    Warm colors indicate positive curvature (echo growing); cool colors
    indicate negative curvature (echo decaying).
    The dashed contour marks the $C_{\mathrm{mem}}=0$ boundary separating
    mid-lag echo growth from mid-lag echo decay.
}
\label{fig:fig2_memory_curvature}
\end{figure}
