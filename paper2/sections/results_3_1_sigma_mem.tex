% Paper 2 - Section 3.1: Universal memory-collapse threshold
% PLACEHOLDER - Kelly to fill

% Key result from Stage 1 / 1b:
% σ_mem ≈ 0.002 is CONSTANT across all K values tested
% This is ~30× lower than σ_c (CHSH collapse threshold)

The first striking result is the existence of a \emph{universal} memory-collapse threshold $\sigma_{\mathrm{mem}}$ that is independent of coupling strength $K$.

Figure~\ref{fig:sigma_mem_vs_sigma_c} compares the memory-collapse threshold $\sigma_{\mathrm{mem}}(K)$ with the CHSH-collapse threshold $\sigma_c(K)$ from Paper~1.

\begin{figure}[H]
    \centering
    % TODO: Insert actual figure
    \includegraphics[width=\textwidth]{figs/placeholder_fig1.pdf}
    \caption{%
        \textbf{Memory collapses ${\sim}30\times$ before CHSH amplitude.}
        (a) $\sigma_{\mathrm{mem}}(K)$ (coral) vs.\ $\sigma_c(K)$ (teal) on log scale. The shaded region is the ``memory-fragile zone'' where $|S|$ remains high but temporal coherence is lost.
        (b) Fragility ratio $r(K) = \sigma_{\mathrm{mem}}/\sigma_c \approx 0.03$--$0.08$ across the ridge.
        % DATA SOURCE: E211, E211b
    }
    \label{fig:sigma_mem_vs_sigma_c}
\end{figure}


\subsubsection{Catastrophic Collapse}

High-resolution zoom (Stage 1b) reveals that the collapse is \emph{catastrophic} rather than gradual:

\begin{quote}
The collapse of long-lag CHSH memory occurs at $\sigma_{\mathrm{mem}} \approx 0.002$ uniformly across all coupling strengths tested. The drop is discontinuous at our resolution: $\rho_S(50)$ plunges from ${\approx}0.95$ at $\sigma=0$ to ${\approx}0.003$ at $\sigma=0.002$. This indicates a catastrophic transition rather than gradual decoherence, consistent with the onset of chaotic phase diffusion dominating the CHSH angle geometry. Notably, the collapse is $K$-independent, suggesting that echo coherence is controlled by a universal fragility mode not visible in $|S|$ itself.
\end{quote}

We verified the collapse location using $\Delta t = 0.001$ and found
$\sigma_{\mathrm{mem}} = 0.002 \pm 0.0005$ across all tested $K$, confirming
that the threshold is not a timestep artifact.

\begin{figure}[H]
    \centering
    % TODO: Insert zoom figure
    \includegraphics[width=0.8\textwidth]{figs/placeholder_fig1b.pdf}
    \caption{%
        \textbf{Step-function memory collapse.}
        $\rho_S(\tau=50)$ vs.\ $\sigma$ for $K \in \{0.3, 0.6, 0.9\}$ in the high-resolution zoom window $\sigma \in [0, 0.04]$. All three curves collapse at $\sigma_{\mathrm{mem}} = 0.002$ from $\rho \approx 0.95$ to $\rho \approx 0$.
        % DATA SOURCE: E211b
    }
    \label{fig:sigma_mem_zoom}
\end{figure}


\subsubsection{Physical Interpretation}

The $K$-independence of $\sigma_{\mathrm{mem}}$ implies that memory fragility is controlled by a \emph{universal} mechanism---likely the onset of chaotic phase diffusion that disrupts the delicate angle geometry underlying the CHSH functional, regardless of how strongly the oscillators are coupled.

This stands in contrast to $\sigma_c$, which scales linearly with $K$ (Paper~1), reflecting the noise level at which synchronization itself breaks down.

Increasing $K$ suppresses global phase diffusion but does not prevent local
twisting of the optimal CHSH geometry. Since CHSH depends on relative phase,
memory loss arises from angular shear rather than absolute phase drift. This
explains why $\sigma_{\mathrm{mem}}$ is insensitive to coupling even though
$\sigma_c(K)$ increases.
