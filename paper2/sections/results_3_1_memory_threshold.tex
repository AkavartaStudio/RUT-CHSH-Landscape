% Paper 2 - Results Section 3.1: Universal Memory Threshold
% Camera-ready draft from Kelly

\subsection{Universal noise threshold for temporal CHSH memory}
\label{sec:universal_memory_threshold}

In Paper~1 we characterized how the CHSH functional $S$ responds to increasing noise,
identifying a coupling-dependent collapse boundary $\sigma_c(K)$ at which the
noise-averaged value $\langle |S| \rangle$ falls below the classical limit
$|S| = 2$. Here we ask a different question: at what noise level does the
\emph{memory} of these correlations disappear, even while the instantaneous
geometry of $S$ remains strongly nonclassical?

To answer this, we use the same driven two-oscillator model and CHSH angle set
as in Paper~1, but focus on the temporal correlation of the CHSH signal.
For each parameter pair $(K,\sigma)$ we compute the lagged autocorrelation
$\rho_S(\tau)$ of the instantaneous CHSH value $S(t)$ and use $\rho_S(\tau=50)$
as a canonical long-lag memory diagnostic. At fixed coupling $K$ we first
estimate the deterministic memory level
$\rho_{\mathrm{det}}(K) = \rho_S(\tau=50; K, \sigma = 0)$, which is typically
$\rho_{\mathrm{det}} \approx 0.95$ along the high-$|S|$ ridge. We then define a
memory-collapse threshold $\sigma_{\mathrm{mem}}(K)$ as the smallest noise
level for which
\begin{equation}
  \rho_S(\tau=50; K, \sigma) \;<\; f\,\rho_{\mathrm{det}}(K),
  \qquad f = 0.5,
  \label{eq:sigma_mem_def}
\end{equation}
i.e.\ the point where long-lag CHSH memory has fallen below half its
deterministic value. This definition is intentionally conservative: it does not
require $\rho_S(\tau)$ to vanish, only to lose a substantial fraction of its
coherence.

A coarse sweep over the parameter range $K \in [0.1,1.0]$ and
$\sigma \in [0.0,0.40]$ with $\Delta\sigma = 0.02$ already reveals a striking
separation of scales. For every coupling tested, the extracted
$\sigma_{\mathrm{mem}}(K)$ is approximately constant:
$\sigma_{\mathrm{mem}}(K) \approx 0.02$, while the CHSH-collapse boundary
$\sigma_c(K)$ inherited from Paper~1 rises from about $0.4$ at weak coupling
to $0.7$--$0.8$ near the ridge. Thus, across the entire high-$|S|$ region,
there exists a broad band $\sigma_{\mathrm{mem}} \lesssim \sigma \lesssim \sigma_c(K)$
in which the instantaneous CHSH value remains large and $|S| > 2$, yet
long-lag temporal memory has already collapsed. In this intermediate regime the
system continues to produce strong snapshot correlations, but no longer
remembers its own correlation history.

A high-resolution zoom around the coarse threshold shows that this separation
is even more extreme than the coarse sweep suggests. For three representative
couplings $K = 0.3, 0.6, 0.9$ we scan $\sigma \in [0.0, 0.04]$ with
$\Delta\sigma = 0.002$. In all cases the long-lag autocorrelation
$\rho_S(\tau=50)$ drops catastrophically from $\approx 0.95$ at $\sigma=0$
to values on the order of $10^{-3}$ at $\sigma = 0.002$, with no gradual
decrease in between at our resolution. Within numerical precision, the
memory-collapse threshold is therefore $\sigma_{\mathrm{mem}} \approx 0.002$
and, notably, is \emph{independent of $K$} across the ridge. By contrast, the
CHSH-collapse scale $\sigma_c(K)$ remains of order $0.5$--$0.8$ and retains its
strong coupling dependence. As a result, the ratio
$\sigma_{\mathrm{mem}}/\sigma_c(K)$ lies at only a few percent throughout the
high-$|S|$ domain.

These results establish two key points. First, temporal CHSH memory is
\emph{exponentially more fragile} to noise than the instantaneous CHSH
geometry: in our system, long-lag correlation memory collapses at a noise
scale roughly $30$ times smaller than the scale required to destroy
$|S|>2$. Second, the memory-collapse threshold is effectively universal along
the ridge: while $\sigma_c(K)$ tracks the details of the coupling, the scale at
which the system forgets its own correlation history is set by an almost
$K$-independent noise level. The $(K,\sigma)$ plane is therefore naturally
partitioned into three regimes: an echo-coherent phase
($\sigma \lesssim \sigma_{\mathrm{mem}}$) in which both $|S|$ and $\rho_S$ are
large; a memory-dead but CHSH-strong phase
($\sigma_{\mathrm{mem}} \lesssim \sigma \lesssim \sigma_c(K)$) in which
$|S|>2$ persists despite the loss of temporal coherence; and a classical phase
($\sigma \gtrsim \sigma_c(K)$) in which both diagnostics have relaxed below the
CHSH bound. The remainder of this paper is devoted to resolving the internal
structure of these regimes using more detailed memory and curvature measures.

The catastrophic nature of the long-lag memory collapse raises a deeper
question: how does the internal structure of the echo evolve before the
collapse? To resolve this internal geometry, we turn to the memory curvature
$C_{\mathrm{mem}}(K,\sigma)$, which captures whether mid-lag memory is
amplifying or decaying. This allows us to see inside the memory layer.


%=============================================================================
% FIGURES
%=============================================================================

\begin{figure}[H]
\centering
\begin{subfigure}[t]{0.9\linewidth}
    \centering
    \includegraphics[width=\linewidth]{figs/fig1_sigma_mem_comparison}
    \caption{Comparison of the long-lag memory-collapse threshold
    $\sigma_{\mathrm{mem}}$ with the CHSH collapse boundary
    $\sigma_{\mathrm{c}}(K)$.
    The left panel shows that $\sigma_{\mathrm{mem}}$ is effectively flat
    across all coupling strengths tested, with
    $\sigma_{\mathrm{mem}} \approx 0.002$ independent of $K$.
    In contrast, the CHSH boundary $\sigma_{\mathrm{c}}(K)$ grows
    approximately linearly with coupling, consistent with the
    empirical relation $\sigma_{\mathrm{c}} \approx 0.60 K + 0.22$ from
    Paper~1.
    The right panel plots the ratio $\sigma_{\mathrm{mem}} / \sigma_{\mathrm{c}}$,
    highlighting the extreme separation of scales:
    long-lag memory collapses roughly $25$--$40\times$ earlier than
    instantaneous CHSH structure across the entire ridge.}
    \label{fig:fig1a_sigma_mem_comparison}
\end{subfigure}

\vspace{1em}

\begin{subfigure}[t]{0.9\linewidth}
    \centering
    \includegraphics[width=\linewidth]{figs/fig1b_sigma_mem_zoom}
    \caption{\textbf{Catastrophic collapse of long-lag CHSH memory.}
    Long-lag autocorrelation $\rho_S(\tau=50)$ as a function of noise strength
    $\sigma$ for three representative couplings $K = 0.3, 0.6, 0.9$.
    In all cases the deterministic system ($\sigma=0$) exhibits strong temporal
    memory with $\rho_S(\tau=50) \approx 0.95$. Introducing a tiny amount of
    noise at $\sigma = 0.002$ causes $\rho_S$ to plummet to values on the
    order of $10^{-3}$, with no gradual decay resolved at intermediate
    $\sigma$. The memory-collapse threshold is thus
    $\sigma_{\mathrm{mem}} \approx 0.002$, independent of $K$ at our
    resolution, confirming that long-lag CHSH memory is extraordinarily
    fragile compared to the instantaneous CHSH geometry.}
    \label{fig:fig1b_sigma_mem_zoom}
\end{subfigure}
\caption{Universal memory-collapse threshold $\sigma_{\mathrm{mem}}$ and its
    relationship to the CHSH collapse boundary $\sigma_{\mathrm{c}}(K)$.}
\label{fig:fig1_memory_threshold}
\end{figure}
