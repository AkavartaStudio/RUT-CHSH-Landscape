% Paper 2 - Results Section 3.4: Synthesis
% Camera-ready draft from Kelly

\subsection{Synthesis: Three regimes of CHSH memory dynamics}
\label{sec:three_regimes}

Taken together, the results above reveal that the CHSH landscape is not a
single, smooth surface but a three--regime structure in which temporal
correlations behave very differently depending on location in the
$(K,\sigma)$ plane. These regimes appear consistently across all diagnostics
evaluated---catastrophic memory thresholds, lag--dependent curvature, long--lag
echo surfaces, and mid--lag susceptibilities---and together provide a coherent
picture of how CHSH memory is organized in classical two--oscillator dynamics.

\paragraph{(1) Basin of trivial memory.}
For noise levels above the collapse threshold $\sigma_{\mathrm{c}}(K)$, the
system enters a decorrelated region in which $|S|<2$, the long--lag echo
$\rho_S(\tau^\ast)$ relaxes to values near zero (with weak negative
excursions), and all curvature diagnostics are dominated by decay. This
``basin'' behaves as a flat, noise--dominated phase: temporal correlations are
featureless, susceptibilities are small in magnitude, and the CHSH functional
itself carries no nontrivial structure. All memory diagnostics reduce to
classical relaxation.

\paragraph{(2) Intermediate band with mid--range echo amplification.}
Below the CHSH collapse curve but above the universal memory threshold
$\sigma_{\mathrm{mem}}\approx 0.002$, we observe a broad band in which
temporal correlations are neither trivial nor robust. Here the instantaneous
CHSH functional remains large ($|S|\gtrsim 2$), yet long--lag memory is already
suppressed. Within this band the curvature analysis identifies a consistent
mid--lag ``bulge'': $C_{\mathrm{mem}}$ is positive at intermediate lag while
remaining predominantly negative at both short and long lags. This indicates a
distinctive, three--segment memory profile---initial decay, mid--range enhancement,
and final relaxation---that does not appear in the basin or on the ridge. The
susceptibility surface similarly shows that this entire region is highly
sensitive to noise: $\chi>0$ across most of the band, with a tangled
$\chi=0$ contour delineating small pockets of local flattening. This band is
therefore a structured ``transitional'' regime where echoes are present but
fragile, shaped by the interplay between coupling and noise.

\paragraph{(3) Ridge of instantaneous coherence without temporal persistence.}
Along the noise--free axis and extending slightly into finite $\sigma$, the
system forms a sharp ridge where the CHSH functional attains its maximal
values ($|S|\approx 2.8$) and the long--lag echo is initially strong. Yet this
ridge lies entirely above the universal memory threshold: even infinitesimal
noise---on the order of $\sigma=0.002$---collapses long--lag memory from
$\rho_S\approx 0.95$ to zero. The ridge therefore supports \emph{instantaneous}
geometric coherence but not \emph{persistent} temporal memory. In this regime
curvature is dominated by negative short-- and long--lag values with at most a
shallow mid--lag hump, and the susceptibility is strongly positive, indicating
that even slight perturbations produce large changes in the correlation
profile.

\paragraph{Summary.}
These three regimes---(i) a noise--dominated basin with trivial memory,
(ii) a structured intermediate band with mid--lag enhancement, and
(iii) a sharp ridge of instantaneous but nonpersistent coherence---together
establish that CHSH memory in classical oscillators is organized not as a
single monotonic decay but as a multi--layered temporal structure. This
three--regime map provides the framework for the global interpretation in
Section~\ref{sec:discussion} and motivates the broader topographic analysis of
observer fields developed in Paper~3.


%=============================================================================
% FIGURE
%=============================================================================

\begin{figure}[H]
\centering
\includegraphics[width=\linewidth]{figs/fig5_angle_field_panel}
\caption{
    Angle-resolved CHSH field across the $(K,\sigma)$ plane.
    For each grid point the measurement angles $(a,a',b,b')$ were numerically
    optimized to maximize the continuous CHSH functional, yielding an optimal
    value $S^*(K,\sigma)$ and corresponding echo at lag $\tau=50$.
    On the coherent ridge the optimal angles remain close to the canonical
    configuration; in the intermediate band they rotate rapidly, signaling
    geometric shear; deep in the noise basin no angle choice recovers $|S|>2$.
}
\label{fig:fig5_angle_field_panel}
\end{figure}
