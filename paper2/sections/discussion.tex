% Paper 2 - Discussion
% Camera-ready draft from Kelly
\label{sec:discussion}

The results above show that temporal structure in CHSH-like classical
oscillators is far richer than suggested by instantaneous correlation
diagnostics alone. While Paper~1 established that the two--oscillator CHSH
functional forms a sharply defined ``ridge'' of high $|S|$ across coupling
strengths, the present study reveals that this ridge has a markedly layered
temporal character: long--lag memory, mid--range curvature, and noise
susceptibility each carve out distinct regions of behavior in the
$(K,\sigma)$ plane.

The most striking observation is the scale separation between the CHSH
collapse threshold $\sigma_{\mathrm{c}}(K)$ and the universal memory threshold
$\sigma_{\mathrm{mem}} \approx 0.002$. Above $\sigma_{\mathrm{c}}(K)$ the system
enters a trivial, noise-dominated basin in which both $|S|$ and all memory
measures decay to classical values. In contrast, the memory threshold lies
two orders of magnitude lower: long--lag CHSH memory collapses almost
instantaneously, far earlier than the CHSH functional itself. This establishes
that \emph{geometric coherence} (as measured by $|S|$) and \emph{temporal
persistence} (as measured by $\rho_S(\tau)$) respond very differently to noise,
even within the same deterministic system.

Between these two thresholds lies a broad and previously uncharacterized
intermediate regime. Here $|S|$ remains above the classical bound while long-lag
memory is already suppressed. Within this region the correlation profile
$\rho_S(\tau)$ acquires a distinctive three--segment shape: an initial decay,
a shallow mid-range enhancement (positive $C_{\mathrm{mem}}$), and eventual
relaxation at long lag. This ``bulged'' temporal structure indicates that
CHSH-like memory is neither monotonic nor featureless, but contains an
intermediate scale at which correlations are transiently reinforced.

The echo and susceptibility surfaces reinforce this picture. The
$\rho_S(\tau^\ast)$ surface exhibits a sharp high--coherence plateau along the
noise--free axis that diminishes rapidly as noise is introduced, while the
susceptibility surface shows that mid-range memory is extraordinarily sensitive
to small perturbations even when the CHSH functional remains robust. The
$\chi=0$ boundary, with roughly ninety crossings across the sampled domain,
suggests that the response of mid-lag memory to noise is structured and
nontrivial, consistent with the layered interpretation above.

Taken together, these results show that the CHSH ridge is not a single wall
but a multi-layered temporal object: instantaneous correlations are robust,
mid-range correlations are fragile but structured, and long-range correlations
collapse at an extremely small noise scale. This decomposition clarifies the
relationship between geometry and dynamics in classical CHSH systems and
provides the conceptual framework for the network-based topographic analysis
developed in Paper~3. Extensions to small oscillator networks build naturally
on the Kuramoto synchronization
literature~\cite{rodrigues2016kuramoto,strogatz2000kuramoto}.
