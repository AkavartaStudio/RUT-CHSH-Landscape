% Paper 2 - Results Section 3.3: Echo and χ Surfaces
% Camera-ready draft from Kelly

\subsection{Echo and susceptibility surfaces}
\label{sec:echo_chi_surfaces}

We next summarize the global structure of CHSH memory across the full
$(K,\sigma)$ plane using two complementary diagnostics: a long-lag echo
surface $\rho_S(\tau^\ast)$ and a mid-lag susceptibility $\chi$, both evaluated
on the same grid of 19 couplings and 21 noise levels.

Figure~\ref{fig:fig3_echo_surface} shows the echo surface
$\rho_S(\tau=50)$ across the $(K,\sigma)$ plane.
At zero noise, the entire high-$|S|$ ridge lies on a broad plateau with
$\rho_S \approx 0.95$, indicating strong temporal persistence of CHSH
correlations.
However, even a small amount of noise disrupts this structure:
at $\sigma \approx 0.002$ the echo collapses sharply across all $K$,
falling to values near zero.
Above this universal threshold the surface remains essentially flat, with
$\rho_S$ indistinguishable from noise-level fluctuations throughout both the
intermediate band and the noise basin.
Thus, echo persistence is not graded across the ridge but instead exhibits a
catastrophic transition: temporal coherence is robust in the deterministic
limit but becomes extremely fragile once noise is introduced, collapsing
roughly thirty times earlier than the instantaneous CHSH functional.

Echo strength shows where memory survives, but not how fragile the CHSH
geometry becomes to angular perturbations. This requires a directional
derivative. We therefore examine the susceptibility $\chi$, which reveals the
shear, instability pockets, and rotational flow within the CHSH field.

While the echo surface captures the absolute level of long-lag memory, it does
not quantify how \emph{sensitive} that memory is to changes in noise. To do
so, we construct a mid-lag susceptibility $\chi(K,\sigma)$ by measuring the
finite-difference response of $\rho_S(\tau_{\mathrm{mid}})$ to small changes in
$\sigma$ at fixed coupling. The resulting $\chi$ surface, shown in
Fig.~\ref{fig:fig4_chi_surface}, is overwhelmingly positive: across the grid,
approximately $86\%$ of sampled points have $\chi>0$, indicating that in most
of the high-$|S|$ and intermediate regions an incremental increase in noise
leads to a disproportionate reduction in mid-lag memory. The remaining
$\sim 14\%$ of points with $\chi<0$ form scattered pockets in which additional
noise locally flattens or suppresses structure in a way that slightly
stabilizes the mid-range correlation.

The $\chi=0$ contour, highlighted in Fig.~\ref{fig:fig4_chi_surface} as a
black curve, marks a nontrivial boundary between echo-amplifying and
echo-suppressing regimes: we identify around ninety distinct crossings of this
contour across the $(K,\sigma)$ grid, confirming that the susceptibility
landscape is far from featureless. In combination with the echo surface, this
reveals that the CHSH ridge is not merely a single ``wall'' of high
instantaneous correlation, but is embedded within a broader band where
long-lag echoes are both strong and extremely sensitive to perturbations. The
system therefore behaves as a kind of classical echo amplifier near the ridge,
with long-lag memory that is simultaneously large in magnitude and highly
fragile to noise.


%=============================================================================
% FIGURES
%=============================================================================

\begin{figure}[H]
\centering
\includegraphics[width=0.9\linewidth]{figs/fig3_echo_surface}
\caption{
    Echo surface $\rho_S(\tau = 50)$ across the $(K,\sigma)$ plane.
    Colors show the lagged CHSH autocorrelation at $\tau=50$ sampling
    intervals. At $\sigma=0$ the system lies on a high-coherence plateau
    with $\rho_S \approx 0.95$ along the entire CHSH ridge. As noise
    increases, echo strength collapses sharply near the universal memory
    threshold $\sigma_{\mathrm{mem}} \approx 0.002$ and remains near zero
    throughout the intermediate band and noise basin, even in regions where
    the instantaneous CHSH functional still satisfies $|S|>2$.
}
\label{fig:fig3_echo_surface}
\end{figure}

\begin{figure}[H]
\centering
\includegraphics[width=0.9\linewidth]{figs/fig4_chi_surface}
\caption{
    Susceptibility surface $\chi(K,\sigma)$ showing the directional
    derivative of the CHSH functional with respect to measurement angle.
    Positive values indicate that small rotations of the measurement
    geometry increase $|S|$, while negative values indicate decreasing
    sensitivity.
    The coherent ridge exhibits a broad region of positive $\chi$,
    reflecting a stable and well-oriented CHSH geometry.
    In the intermediate band, $\chi$ alternates sign and develops shear
    structures, indicating competition between nearby angle optima and the
    onset of rotational flow in the CHSH field.
    Deep in the noise basin, $\chi \approx 0$, consistent with the loss of
    all directional CHSH structure when $|S|<2$ for every angle choice.
}
\label{fig:fig4_chi_surface}
\end{figure}

Susceptibility shows where the CHSH geometry bends, but not how the optimal
measurement directions reorganize across the field. To expose the underlying
geometric structure, we compute the angle-resolved field geometry from
Experiment E231, which reveals the rotational shear and the distinct regimes
of CHSH organization.
