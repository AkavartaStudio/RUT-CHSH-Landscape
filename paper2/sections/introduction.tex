% Paper 2 - Introduction
% Camera-ready draft from Kelly

\subsection{Motivation}

Coupled phase oscillators provide a simple but expressive setting for studying
CHSH-like correlations in classical dynamics. In Paper~1 we established that a
pair of deterministically coupled oscillators produces a well-defined CHSH
landscape: a high-$|S|$ ridge for sufficiently strong coupling and low noise,
and a sharp collapse curve $\sigma_{\mathrm{c}}(K)$ that separates high-$|S|$
from classical regimes. Yet that analysis treated the CHSH functional as a
time-averaged quantity, collapsing the temporal dimension into a single
statistic.

Here we restore that temporal dimension. By computing lag-resolved correlations
across the full $(K,\sigma)$ plane, we reveal the internal structure of the
CHSH ridge: how $S(t)$ ``remembers'' itself at finite lags, how that memory
degrades under noise, and whether the system's temporal coherence can survive
when the instantaneous magnitude $|S|$ remains high.

\subsection{Background and Related Work}

The CHSH inequality~\cite{clauser1969proposed} and its
generalizations~\cite{cirelson1980quantum} have been extensively studied in
the context of quantum foundations and information theory, building on Bell's
original work~\cite{bell1964einstein}. Classical analogues
of Bell-type correlations arise in a variety of dynamical systems, from coupled
oscillators to chaotic maps, whenever continuous measurement parameters can be
tuned to maximize correlations between subsystems. Following the classical Kuramoto
framework~\cite{kuramoto1984chemical,strogatz2000kuramoto,acebron2005kuramoto},
we adopt a minimal setting in which coupling strength and noise amplitude
can be varied independently, allowing systematic exploration of the
correlation landscape.

Previous work has focused primarily on instantaneous or time-averaged CHSH
statistics. The temporal structure of correlations---how CHSH memory decays
with lag, how curvature varies across parameter space, and how susceptibility
responds to noise---has received less attention. This paper addresses that gap
by introducing lag-resolved diagnostics that reveal the internal temporal
organization of the CHSH ridge.

\subsection{Contributions}

The central finding is that instantaneous magnitude and temporal memory are
governed by distinct noise scales. CHSH memory collapses at
$\sigma_{\mathrm{mem}} \approx 0.002$---roughly thirty times lower than the
$|S|$-collapse curve $\sigma_{\mathrm{c}}(K)$. This separation defines three
robust regimes in the $(K,\sigma)$ plane: (1) a noise-dominated basin where
both $|S|$ and memory are trivial; (2) a structured intermediate band where
$|S|>2$ persists but long-lag memory has already vanished; and (3) a narrow
ridge of maximal $|S|$ and non-zero memory. Throughout these regimes, the
curvature and susceptibility of the CHSH memory field encode nontrivial
temporal structure that cannot be read from the $|S|$ functional alone.

We further examine the CHSH functional as a directional field by scanning
across all measurement geometries. This angle-resolved analysis shows that the
optimal CHSH configuration twists under noise, with the susceptibility
$\partial S^*/\partial\sigma$ changing sign multiple times as coupling and
noise vary. Evaluated at these optimal angles, long-lag memory still collapses
at $\sigma_{\mathrm{mem}}$, confirming that the memory threshold arises from
intrinsic dynamics rather than geometric artifact.

\subsection{Outline}

The paper is organized as follows. Section~2 defines the observables---echo
strength $\rho_S(\tau)$, memory curvature, and susceptibility---and specifies
the simulation and grid parameters. Section~3 presents results in five
subsections: (3.1) the $\sigma_{\mathrm{mem}}$ threshold and regime
separation; (3.2) the curvature surface and its sign structure; (3.3) echo
and susceptibility surfaces; (3.4) synthesis of the three regimes; and
(3.5) angle-resolved observer fields. Section~4
discusses implications for classical CHSH dynamics and the coherence-memory
distinction. The conclusion summarizes the framework and sets the stage for
Paper~3, where we extend the analysis to small oscillator networks.
