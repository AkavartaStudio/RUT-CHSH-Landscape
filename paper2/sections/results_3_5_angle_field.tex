% Paper 2 - Section 3.5: Angle-Resolved Geometry of the CHSH Field
% Camera-ready draft from Kelly

\subsection{Angle-Resolved Geometry of the CHSH Field}
\label{sec:angle_field}

The previous sections established that CHSH structure in the two-oscillator
system is not a single quantity but a multi-layered field with distinct
temporal components. Instantaneous correlations remain strong across the
high-$|S|$ ridge, whereas long-lag persistence collapses at a universal noise
threshold $\sigma_{\mathrm{mem}}\approx 0.002$. Curvature and susceptibility
maps revealed additional structure, including sign changes in the intermediate
band and a sharp coherent spine along the ridge. Together these results suggest
that the CHSH functional, treated as a directional probe, must itself vary
meaningfully with measurement geometry.

In this section we characterize the \emph{angle-resolved CHSH field} across the
$(K,\sigma)$ plane. For each grid point we scan over all measurement geometries
$(a,a',b,b')$ and identify both the optimal angles that maximize $|S|$ and the
local orientation of the CHSH gradient. This analysis transforms the CHSH
surface from a scalar field into a four-dimensional directional object whose
shape reflects how the system responds to perturbations in measurement
geometry.

The key advantage of this approach is that angle dependence reacts more
sensitively to noise than the absolute value of $|S|$. Near the coherent ridge
the optimal angles remain close to the canonical $(0^\circ,98^\circ,45^\circ,
127^\circ)$ configuration used in Paper~1. As noise increases the optimal
angles twist systematically, forming a smooth rotational flow that predicts the
location of the collapse boundary. Inside the intermediate band, where long-lag
memory has already failed but instantaneous coherence survives, the angle field
develops shear and local vortices indicative of competing phase preferences.
Deep inside the noise basin, the angle field degenerates, with the optimal
angles drifting freely and $|S|<2$ for all configurations.

This angle-resolved view completes the geometric decomposition of the CHSH
landscape. Instead of a single ``ridge,'' the system exhibits: (i) a coherent
spine where optimal geometry is stable; (ii) a twisting transition region where
noise induces rotation and shear in the angle field; and (iii) a noise basin in
which directional structure collapses entirely.

% Figure moved to results_3_4_chi.tex per patch instructions

Figure~\ref{fig:fig5_angle_field_panel} summarizes the angle-resolved CHSH field
obtained in Experiment E231. For each point in the $(K,\sigma)$ plane we
numerically optimized the measurement angles to maximize the continuous
CHSH functional, yielding an optimal value $S^*(K,\sigma)$ and a
corresponding echo value at lag $\tau=50$. Across the full grid the
optimized CHSH surface spans the range $1.45 \lesssim S^* \lesssim 2.83$
with a mean of $S^* \approx 2.71$: deep inside the noise basin even the
best measurement geometry cannot recover a violation ($S^*<2$), whereas on
the coherent ridge the optimized values remain close to the algebraic
maximum. At $\sigma=0$ the optimal-angle configurations preserve perfect
temporal memory with $\rho_S(\tau) = 1.0$, but away from this deterministic
limit the echo at the optimal geometry degrades, taking values as low as
$\rho_S \approx 0.43$.

The optimized angles themselves vary substantially across the field: the
best setting $a$ spans approximately $29^\circ$--$179^\circ$ and $b'$ spans
$13^\circ$--$186^\circ$, indicating strong twisting of the preferred
measurement geometry as $(K,\sigma)$ changes. On the coherent ridge the
optimal angles remain close to the canonical CHSH configuration, but in the
intermediate band they rotate rapidly, signaling geometric shear and
competing local optima in the CHSH field. These patterns reinforce the
three-regime picture: a well-aligned ridge with high $S^*$ and strong
memory, a geometrically unstable intermediate band with reduced echo, and a
noise basin in which CHSH structure cannot be recovered by any angle
choice.
