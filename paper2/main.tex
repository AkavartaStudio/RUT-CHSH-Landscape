\documentclass[11pt,a4paper]{article}
\usepackage[utf8]{inputenc}
\usepackage{amsmath,amssymb,amsfonts}
\usepackage{graphicx}
\usepackage{hyperref}
\usepackage{cleveref}
\usepackage[margin=1in]{geometry}
\usepackage{mathptmx}
\usepackage{float}
\usepackage[font=small,labelfont=bf]{caption}
\usepackage{subcaption}
\usepackage{parskip}
\setlength{\parindent}{15pt}
\setlength{\parskip}{0pt}
\usepackage{physics}
\usepackage{titlesec}
\usepackage{tocloft}

% Equation spacing
\setlength{\abovedisplayskip}{10pt}
\setlength{\belowdisplayskip}{10pt}

% Section header formatting
\titleformat{\section}{\normalsize\normalfont\bfseries\Large}{\thesection}{1em}{}[\normalsize]
\titleformat{\subsection}{\normalsize\normalfont\bfseries\large}{\thesubsection}{1em}{}[\normalsize]
\titleformat{\subsubsection}{\normalsize\normalfont\bfseries}{\thesubsubsection}{1em}{}[\normalsize]

% TOC formatting
\cftsetpnumwidth{2em}
\cftsetrmarg{3em}

\title{Echo Geometry of the Classical CHSH Ridge:\\
Memory, Curvature, and Directional Susceptibility\\[10pt]
\large Paper 2: The Memory Landscape}
\author{Kelly McRae}
\date{December 2025}

\hypersetup{
    colorlinks=true,
    linkcolor=blue,
    citecolor=blue,
    urlcolor=blue
}

\begin{document}
\normalfont
\maketitle

\begin{abstract}
Paper~1 showed that a pair of deterministically coupled oscillators exhibits a
well-defined CHSH landscape: a high-$|S|$ ridge at low noise and a sharp
collapse boundary at $\sigma_{\mathrm{c}}(K)$. Here we extend this analysis by
resolving the \emph{temporal structure} of CHSH correlations across the full
$(K,\sigma)$ plane. Using lag-resolved correlation decay, curvature,
susceptibility, and angle-dependent scanning, we identify three robust regimes.
(1) A trivial noise-dominated basin in which $|S|<2$ and all memory measures
vanish. (2) A structured intermediate band where $|S|>2$ persists but
long-lag memory has already collapsed, marked by a shallow mid-range
enhancement in $\rho_S(\tau)$ and a complex pattern of positive and negative
curvature. (3) A narrow ridge of strong instantaneous coherence in which
$|S|\approx 2.8$ but temporal persistence remains extremely fragile. Across the
entire domain we find a universal memory threshold
$\sigma_{\mathrm{mem}}\approx 0.002$, two orders of magnitude below the CHSH
collapse curve, beyond which long-lag CHSH memory cannot be sustained for any
choice of angles, confirming that memory collapse reflects intrinsic
relative-phase dynamics rather than discretization or finite sampling.

We further treat the CHSH functional as a directional field by scanning across
all measurement geometries. The resulting angle-resolved surfaces show that
the optimal CHSH configuration twists under noise, exhibiting multiple
susceptibility sign changes and a well-defined angle-flow structure. Evaluated
at these optimal angles, long-lag memory still collapses at the universal
threshold, confirming that temporal coherence is limited by intrinsic dynamics
rather than geometry. Together, these results show that the CHSH ridge is a
multi-layered temporal object whose instantaneous and lagged correlations are
governed by distinct noise scales. This temporal decomposition provides the
framework used in Paper~3 to analyze CHSH-like memory in small oscillator
networks.
\end{abstract}

\newpage
\tableofcontents
\newpage


%=============================================================================
\section{Introduction}
%=============================================================================

% Paper 3 - Introduction
% Scaffold - to be written

% PLACEHOLDER

The two-oscillator system studied in Papers~1 and~2 provides the simplest
setting for CHSH-like correlations in classical dynamics. Paper~1 mapped the
instantaneous CHSH landscape across the $(K,\sigma)$ plane, identifying a
high-$|S|$ ridge and a noise-dependent collapse boundary. Paper~2 resolved the
temporal structure of this landscape, revealing a universal memory threshold
$\sigma_{\mathrm{mem}}\approx 0.002$ that is orders of magnitude below the
instantaneous collapse scale.

These results raise a natural question: how does network topology shape CHSH
memory? In this paper we extend the analysis to three-oscillator systems,
studying three canonical motifs---chain, star, and triangle---that capture
distinct connectivity patterns. Each topology defines a different coupling
graph and therefore a different pathway for correlation flow.

% TODO: Outline the three topologies
% TODO: State the main questions
% TODO: Preview results



%=============================================================================
\section{Methods}
%=============================================================================

% Paper 3 - Methods
% Scaffold - to be written

% PLACEHOLDER

\subsection{Three-oscillator model}
\label{sec:three_osc_model}

We extend the driven two-oscillator system of Papers~1 and~2 to three coupled
oscillators. The equations of motion take the form
\begin{equation}
  \ddot{x}_i + \omega_i^2 x_i = \sum_{j \neq i} K_{ij}(x_j - x_i) + \sigma\,\xi_i(t),
  \label{eq:three_osc}
\end{equation}
where $K_{ij}$ encodes the coupling topology and $\xi_i(t)$ is independent
Gaussian white noise on each oscillator.

% TODO: Define the three topologies (chain, star, triangle)
% TODO: Specify coupling matrices
% TODO: Define CHSH functional for three-oscillator pairs


\subsection{Topology definitions}
\label{sec:topology_defs}

% Chain: 1--2--3
% Star: 2 central, 1 and 3 peripheral
% Triangle: all-to-all

% TODO: Coupling matrices for each


\subsection{CHSH measurement protocol}
\label{sec:chsh_protocol}

% TODO: Describe how CHSH is computed for each oscillator pair
% TODO: Define echo, curvature, and susceptibility for networks


\subsection{Simulation parameters}
\label{sec:sim_params}

% TODO: Grid specifications
% TODO: Integration scheme
% TODO: Ensemble statistics


%=============================================================================
% EXPERIMENTAL OBJECTIVES (Mission 1)
%=============================================================================

\subsection{Experimental objectives}
\label{sec:experimental_objectives}

The primary objective of Mission~1 is to determine how loop curvature $K$
controls the regime structure of a classical triangular oscillator network, and
to identify the critical value $K_c$ at which the frustrated phase loses global
dominance. Specifically, we aim to:

\begin{enumerate}
  \item \textbf{Detect and localize phase transition(s)} between frustrated and
    flat winding states under curvature sweeps;

  \item \textbf{Quantify detuning effects} on regime frequencies, establishing
    whether asymmetry in edge couplings produces systematic changes in
    frustration;

  \item \textbf{Characterize multi-basin behavior} by determining whether
    multiple attractors coexist at fixed $K$, rather than a single phase
    dominating all initial conditions;

  \item \textbf{Separate existence vs.\ selection}, by distinguishing which
    regimes are permitted by the dynamics (existence) from which regimes are
    actually realized under varying initialization manifolds (selection);

  \item \textbf{Establish reproducibility} across initialization families,
    confirming that observed transitions reflect intrinsic loop dynamics rather
    than artifacts of preparation or numerical procedure.
\end{enumerate}

Together, these objectives probe whether classical triangular loops admit
multiple stable regimes under deterministic dynamics, and whether curvature
alone suffices to predict observed outcomes, or whether preparation history must
be treated as a distinct physical parameter.



%=============================================================================
\section{Results}
%=============================================================================

% Section 3.1 - Camera-ready draft
% Paper 2 - Results Section 3.1: Universal Memory Threshold
% Camera-ready draft from Kelly

\subsection{Universal noise threshold for temporal CHSH memory}
\label{sec:universal_memory_threshold}

In Paper~1 we characterized how the CHSH functional $S$ responds to increasing noise,
identifying a coupling-dependent collapse boundary $\sigma_c(K)$ at which the
noise-averaged value $\langle |S| \rangle$ falls below the classical limit
$|S| = 2$. Here we ask a different question: at what noise level does the
\emph{memory} of these correlations disappear, even while the instantaneous
geometry of $S$ remains strongly nonclassical?

To answer this, we use the same driven two-oscillator model and CHSH angle set
as in Paper~1, but focus on the temporal correlation of the CHSH signal.
For each parameter pair $(K,\sigma)$ we compute the lagged autocorrelation
$\rho_S(\tau)$ of the instantaneous CHSH value $S(t)$ and use $\rho_S(\tau=50)$
as a canonical long-lag memory diagnostic. At fixed coupling $K$ we first
estimate the deterministic memory level
$\rho_{\mathrm{det}}(K) = \rho_S(\tau=50; K, \sigma = 0)$, which is typically
$\rho_{\mathrm{det}} \approx 0.95$ along the high-$|S|$ ridge. We then define a
memory-collapse threshold $\sigma_{\mathrm{mem}}(K)$ as the smallest noise
level for which
\begin{equation}
  \rho_S(\tau=50; K, \sigma) \;<\; f\,\rho_{\mathrm{det}}(K),
  \qquad f = 0.5,
  \label{eq:sigma_mem_def}
\end{equation}
i.e.\ the point where long-lag CHSH memory has fallen below half its
deterministic value. This definition is intentionally conservative: it does not
require $\rho_S(\tau)$ to vanish, only to lose a substantial fraction of its
coherence.

A coarse sweep over the parameter range $K \in [0.1,1.0]$ and
$\sigma \in [0.0,0.40]$ with $\Delta\sigma = 0.02$ already reveals a striking
separation of scales. For every coupling tested, the extracted
$\sigma_{\mathrm{mem}}(K)$ is approximately constant:
$\sigma_{\mathrm{mem}}(K) \approx 0.02$, while the CHSH-collapse boundary
$\sigma_c(K)$ inherited from Paper~1 rises from about $0.4$ at weak coupling
to $0.7$--$0.8$ near the ridge. Thus, across the entire high-$|S|$ region,
there exists a broad band $\sigma_{\mathrm{mem}} \lesssim \sigma \lesssim \sigma_c(K)$
in which the instantaneous CHSH value remains large and $|S| > 2$, yet
long-lag temporal memory has already collapsed. In this intermediate regime the
system continues to produce strong snapshot correlations, but no longer
remembers its own correlation history.

A high-resolution zoom around the coarse threshold shows that this separation
is even more extreme than the coarse sweep suggests. For three representative
couplings $K = 0.3, 0.6, 0.9$ we scan $\sigma \in [0.0, 0.04]$ with
$\Delta\sigma = 0.002$. In all cases the long-lag autocorrelation
$\rho_S(\tau=50)$ drops catastrophically from $\approx 0.95$ at $\sigma=0$
to values on the order of $10^{-3}$ at $\sigma = 0.002$, with no gradual
decrease in between at our resolution. Within numerical precision, the
memory-collapse threshold is therefore $\sigma_{\mathrm{mem}} \approx 0.002$
and, notably, is \emph{independent of $K$} across the ridge. By contrast, the
CHSH-collapse scale $\sigma_c(K)$ remains of order $0.5$--$0.8$ and retains its
strong coupling dependence. As a result, the ratio
$\sigma_{\mathrm{mem}}/\sigma_c(K)$ lies at only a few percent throughout the
high-$|S|$ domain.

These results establish two key points. First, temporal CHSH memory is
\emph{exponentially more fragile} to noise than the instantaneous CHSH
geometry: in our system, long-lag correlation memory collapses at a noise
scale roughly $30$ times smaller than the scale required to destroy
$|S|>2$. Second, the memory-collapse threshold is effectively universal along
the ridge: while $\sigma_c(K)$ tracks the details of the coupling, the scale at
which the system forgets its own correlation history is set by an almost
$K$-independent noise level. The $(K,\sigma)$ plane is therefore naturally
partitioned into three regimes: an echo-coherent phase
($\sigma \lesssim \sigma_{\mathrm{mem}}$) in which both $|S|$ and $\rho_S$ are
large; a memory-dead but CHSH-strong phase
($\sigma_{\mathrm{mem}} \lesssim \sigma \lesssim \sigma_c(K)$) in which
$|S|>2$ persists despite the loss of temporal coherence; and a classical phase
($\sigma \gtrsim \sigma_c(K)$) in which both diagnostics have relaxed below the
CHSH bound. The remainder of this paper is devoted to resolving the internal
structure of these regimes using more detailed memory and curvature measures.

The catastrophic nature of the long-lag memory collapse raises a deeper
question: how does the internal structure of the echo evolve before the
collapse? To resolve this internal geometry, we turn to the memory curvature
$C_{\mathrm{mem}}(K,\sigma)$, which captures whether mid-lag memory is
amplifying or decaying. This allows us to see inside the memory layer.


%=============================================================================
% FIGURES
%=============================================================================

\begin{figure}[H]
\centering
\begin{subfigure}[t]{0.9\linewidth}
    \centering
    \includegraphics[width=\linewidth]{figs/fig1_sigma_mem_comparison}
    \caption{Comparison of the long-lag memory-collapse threshold
    $\sigma_{\mathrm{mem}}$ with the CHSH collapse boundary
    $\sigma_{\mathrm{c}}(K)$.
    The left panel shows that $\sigma_{\mathrm{mem}}$ is effectively flat
    across all coupling strengths tested, with
    $\sigma_{\mathrm{mem}} \approx 0.002$ independent of $K$.
    In contrast, the CHSH boundary $\sigma_{\mathrm{c}}(K)$ grows
    approximately linearly with coupling, consistent with the
    empirical relation $\sigma_{\mathrm{c}} \approx 0.60 K + 0.22$ from
    Paper~1.
    The right panel plots the ratio $\sigma_{\mathrm{mem}} / \sigma_{\mathrm{c}}$,
    highlighting the extreme separation of scales:
    long-lag memory collapses roughly $25$--$40\times$ earlier than
    instantaneous CHSH structure across the entire ridge.}
    \label{fig:fig1a_sigma_mem_comparison}
\end{subfigure}

\vspace{1em}

\begin{subfigure}[t]{0.9\linewidth}
    \centering
    \includegraphics[width=\linewidth]{figs/fig1b_sigma_mem_zoom}
    \caption{\textbf{Catastrophic collapse of long-lag CHSH memory.}
    Long-lag autocorrelation $\rho_S(\tau=50)$ as a function of noise strength
    $\sigma$ for three representative couplings $K = 0.3, 0.6, 0.9$.
    In all cases the deterministic system ($\sigma=0$) exhibits strong temporal
    memory with $\rho_S(\tau=50) \approx 0.95$. Introducing a tiny amount of
    noise at $\sigma = 0.002$ causes $\rho_S$ to plummet to values on the
    order of $10^{-3}$, with no gradual decay resolved at intermediate
    $\sigma$. The memory-collapse threshold is thus
    $\sigma_{\mathrm{mem}} \approx 0.002$, independent of $K$ at our
    resolution, confirming that long-lag CHSH memory is extraordinarily
    fragile compared to the instantaneous CHSH geometry.}
    \label{fig:fig1b_sigma_mem_zoom}
\end{subfigure}
\caption{Universal memory-collapse threshold $\sigma_{\mathrm{mem}}$ and its
    relationship to the CHSH collapse boundary $\sigma_{\mathrm{c}}(K)$.}
\label{fig:fig1_memory_threshold}
\end{figure}



% Section 3.2 - Camera-ready draft
% Paper 2 - Results Section 3.2: Memory Curvature Surface
% Camera-ready draft from Kelly

\subsection{Memory curvature across the CHSH landscape}
\label{sec:memory_curvature}

The analysis above identifies a sharp separation of scales between the noise
level required to destroy long-lag CHSH memory and the much larger scale
required to collapse the CHSH functional itself. To resolve the internal
structure of this ``memory layer'' we now examine how the temporal profile of
$\rho_S(\tau)$ bends as a function of lag and location in the $(K,\sigma)$
plane.

For each parameter pair $(K,\sigma)$ we evaluate the lagged CHSH
autocorrelation $\rho_S(\tau)$ on a small grid of lags
$\tau \in \{\tau_1,\tau_2,\tau_3,\tau_4\}$ and construct a discrete curvature
diagnostic $C_{\mathrm{mem}}(\tau_{\mathrm{mid}}; K,\sigma)$ at three midpoints
$\tau_{\mathrm{mid}} \in \{17.5, 37.5, 75.0\}$ using a second-order finite
difference across neighboring lags. Positive $C_{\mathrm{mem}}$ indicates that
the mid-lag echo lies above the linear interpolation between shorter and longer
lags (locally ``bulging up''), while negative $C_{\mathrm{mem}}$ indicates a
concave profile in which the mid-lag correlation is suppressed relative to a
linear decay.

Figure~\ref{fig:fig2_memory_curvature} summarizes the curvature statistics
across 1995 simulations (19 couplings, 21 noise levels, 5 seeds per point). At
short lag ($\tau_{\mathrm{mid}} = 17.5$), the curvature is predominantly
negative, with $C_{\mathrm{mem}}$ ranging from $-0.031$ to $+0.001$ and a
strong imbalance towards decay-dominated profiles (81 positive, 318 negative
cases). Very close to the present, CHSH memory therefore behaves as a
conventionally relaxing signal: once noise is introduced, the correlation
typically begins bending downward immediately.

At intermediate lag ($\tau_{\mathrm{mid}} = 37.5$), the sign distribution
reverses. Here $C_{\mathrm{mem}}$ remains modest in magnitude
($[-0.005, +0.007]$) but is overwhelmingly positive in sign
(339 positive, 60 negative). Across most of the high-$|S|$ region the mid-lag
echo sits \emph{above} the straight-line interpolation between shorter and
longer lags: as a function of $\tau$, the correlation profile develops a
shallow hump rather than a simple exponential or power-law decay. In other
words, there is a robust ``mid-range'' band in which CHSH memory is slightly
enhanced relative to the naive expectation from its short- and long-lag
behavior.

At long lag ($\tau_{\mathrm{mid}} = 75.0$), the sign distribution flips back
again: $C_{\mathrm{mem}}$ lies in $[-0.001, +0.005]$ with negative values now
dominant (43 positive, 356 negative). Far out in time, the system behaves as a
genuinely decaying process: whatever mid-range enhancement exists has been
exhausted, and the residual correlations drift back toward zero or slightly
negative values.

Taken together, these curvature results show that CHSH memory is not a
featureless exponential tail. Instead, the $(K,\sigma)$ plane contains a broad
intermediate band in which mid-range lags display a slight echo ``bulge,''
meaning that correlations are transiently stronger than suggested by both
earlier and later lags. This intermediate region is embedded between short-
and long-lag regimes in which curvature is dominantly negative. The resulting
three-zone structure is invisible to instantaneous diagnostics such as $|S|$
itself, and only becomes apparent once we track how the correlation profile
bends as a function of lag.

In addition to the broad sign structure, the curvature surface
$C_{\mathrm{mem}}(K,\sigma)$ exhibits small, isolated pockets of negative
curvature embedded within regions of positive curvature. These ``memory
hollows'' occur near the boundary between the coherent ridge and the
intermediate band: short- and long-lag correlations remain nontrivial, but the
mid-lag statistic dips below its neighbors, producing a local inversion of the
echo profile. This indicates that, in these narrow parameter zones, temporal
coherence briefly weakens at intermediate lag before recovering, revealing
fine-grained structure in the way CHSH memory is redistributed as noise is
introduced.

Curvature reveals how memory bends locally in $\tau$, but does not show how
echo behaves globally across the entire $(K,\sigma)$ field. To map this full
geometry, we next examine the echo surface $\rho_S(\tau=50)$, which exposes
the global memory plateau and its abrupt destruction.

The mid-lag enhancement occurs near $\tau \approx O(1/\lambda)$, consistent with
a transient coherence resonance where weak noise reinforces correlations before
eventual long-lag decay.


%=============================================================================
% FIGURE
%=============================================================================

\begin{figure}[H]
\centering
\includegraphics[width=0.9\linewidth]{figs/fig2_memory_curvature}
\caption{
    Memory curvature surface $C_{\mathrm{mem}}(K,\sigma)$ at mid-lag
    ($\tau_{\mathrm{mid}}=37.5$).
    Warm colors indicate positive curvature (echo growing); cool colors
    indicate negative curvature (echo decaying).
    The dashed contour marks the $C_{\mathrm{mem}}=0$ boundary separating
    mid-lag echo growth from mid-lag echo decay.
}
\label{fig:fig2_memory_curvature}
\end{figure}



% Section 3.3 - Camera-ready draft (includes echo + χ surfaces)
% Paper 2 - Results Section 3.3: Echo and χ Surfaces
% Camera-ready draft from Kelly

\subsection{Echo and susceptibility surfaces}
\label{sec:echo_chi_surfaces}

We next summarize the global structure of CHSH memory across the full
$(K,\sigma)$ plane using two complementary diagnostics: a long-lag echo
surface $\rho_S(\tau^\ast)$ and a mid-lag susceptibility $\chi$, both evaluated
on the same grid of 19 couplings and 21 noise levels.

Figure~\ref{fig:fig3_echo_surface} shows the echo surface
$\rho_S(\tau=50)$ across the $(K,\sigma)$ plane.
At zero noise, the entire high-$|S|$ ridge lies on a broad plateau with
$\rho_S \approx 0.95$, indicating strong temporal persistence of CHSH
correlations.
However, even a small amount of noise disrupts this structure:
at $\sigma \approx 0.002$ the echo collapses sharply across all $K$,
falling to values near zero.
Above this universal threshold the surface remains essentially flat, with
$\rho_S$ indistinguishable from noise-level fluctuations throughout both the
intermediate band and the noise basin.
Thus, echo persistence is not graded across the ridge but instead exhibits a
catastrophic transition: temporal coherence is robust in the deterministic
limit but becomes extremely fragile once noise is introduced, collapsing
roughly thirty times earlier than the instantaneous CHSH functional.

Echo strength shows where memory survives, but not how fragile the CHSH
geometry becomes to angular perturbations. This requires a directional
derivative. We therefore examine the susceptibility $\chi$, which reveals the
shear, instability pockets, and rotational flow within the CHSH field.

While the echo surface captures the absolute level of long-lag memory, it does
not quantify how \emph{sensitive} that memory is to changes in noise. To do
so, we construct a mid-lag susceptibility $\chi(K,\sigma)$ by measuring the
finite-difference response of $\rho_S(\tau_{\mathrm{mid}})$ to small changes in
$\sigma$ at fixed coupling. The resulting $\chi$ surface, shown in
Fig.~\ref{fig:fig4_chi_surface}, is overwhelmingly positive: across the grid,
approximately $86\%$ of sampled points have $\chi>0$, indicating that in most
of the high-$|S|$ and intermediate regions an incremental increase in noise
leads to a disproportionate reduction in mid-lag memory. The remaining
$\sim 14\%$ of points with $\chi<0$ form scattered pockets in which additional
noise locally flattens or suppresses structure in a way that slightly
stabilizes the mid-range correlation.

The $\chi=0$ contour, highlighted in Fig.~\ref{fig:fig4_chi_surface} as a
black curve, marks a nontrivial boundary between echo-amplifying and
echo-suppressing regimes: we identify around ninety distinct crossings of this
contour across the $(K,\sigma)$ grid, confirming that the susceptibility
landscape is far from featureless. In combination with the echo surface, this
reveals that the CHSH ridge is not merely a single ``wall'' of high
instantaneous correlation, but is embedded within a broader band where
long-lag echoes are both strong and extremely sensitive to perturbations. The
system therefore behaves as a kind of classical echo amplifier near the ridge,
with long-lag memory that is simultaneously large in magnitude and highly
fragile to noise.


%=============================================================================
% FIGURES
%=============================================================================

\begin{figure}[H]
\centering
\includegraphics[width=0.9\linewidth]{figs/fig3_echo_surface}
\caption{
    Echo surface $\rho_S(\tau = 50)$ across the $(K,\sigma)$ plane.
    Colors show the lagged CHSH autocorrelation at $\tau=50$ sampling
    intervals. At $\sigma=0$ the system lies on a high-coherence plateau
    with $\rho_S \approx 0.95$ along the entire CHSH ridge. As noise
    increases, echo strength collapses sharply near the universal memory
    threshold $\sigma_{\mathrm{mem}} \approx 0.002$ and remains near zero
    throughout the intermediate band and noise basin, even in regions where
    the instantaneous CHSH functional still satisfies $|S|>2$.
}
\label{fig:fig3_echo_surface}
\end{figure}

\begin{figure}[H]
\centering
\includegraphics[width=0.9\linewidth]{figs/fig4_chi_surface}
\caption{
    Susceptibility surface $\chi(K,\sigma)$ showing the directional
    derivative of the CHSH functional with respect to measurement angle.
    Positive values indicate that small rotations of the measurement
    geometry increase $|S|$, while negative values indicate decreasing
    sensitivity.
    The coherent ridge exhibits a broad region of positive $\chi$,
    reflecting a stable and well-oriented CHSH geometry.
    In the intermediate band, $\chi$ alternates sign and develops shear
    structures, indicating competition between nearby angle optima and the
    onset of rotational flow in the CHSH field.
    Deep in the noise basin, $\chi \approx 0$, consistent with the loss of
    all directional CHSH structure when $|S|<2$ for every angle choice.
}
\label{fig:fig4_chi_surface}
\end{figure}

Susceptibility shows where the CHSH geometry bends, but not how the optimal
measurement directions reorganize across the field. To expose the underlying
geometric structure, we compute the angle-resolved field geometry from
Experiment E231, which reveals the rotational shear and the distinct regimes
of CHSH organization.



% Section 3.4 - Synthesis: Three Regimes
% Paper 2 - Results Section 3.4: Synthesis
% Camera-ready draft from Kelly

\subsection{Synthesis: Three regimes of CHSH memory dynamics}
\label{sec:three_regimes}

Taken together, the results above reveal that the CHSH landscape is not a
single, smooth surface but a three--regime structure in which temporal
correlations behave very differently depending on location in the
$(K,\sigma)$ plane. These regimes appear consistently across all diagnostics
evaluated---catastrophic memory thresholds, lag--dependent curvature, long--lag
echo surfaces, and mid--lag susceptibilities---and together provide a coherent
picture of how CHSH memory is organized in classical two--oscillator dynamics.

\paragraph{(1) Basin of trivial memory.}
For noise levels above the collapse threshold $\sigma_{\mathrm{c}}(K)$, the
system enters a decorrelated region in which $|S|<2$, the long--lag echo
$\rho_S(\tau^\ast)$ relaxes to values near zero (with weak negative
excursions), and all curvature diagnostics are dominated by decay. This
``basin'' behaves as a flat, noise--dominated phase: temporal correlations are
featureless, susceptibilities are small in magnitude, and the CHSH functional
itself carries no nontrivial structure. All memory diagnostics reduce to
classical relaxation.

\paragraph{(2) Intermediate band with mid--range echo amplification.}
Below the CHSH collapse curve but above the universal memory threshold
$\sigma_{\mathrm{mem}}\approx 0.002$, we observe a broad band in which
temporal correlations are neither trivial nor robust. Here the instantaneous
CHSH functional remains large ($|S|\gtrsim 2$), yet long--lag memory is already
suppressed. Within this band the curvature analysis identifies a consistent
mid--lag ``bulge'': $C_{\mathrm{mem}}$ is positive at intermediate lag while
remaining predominantly negative at both short and long lags. This indicates a
distinctive, three--segment memory profile---initial decay, mid--range enhancement,
and final relaxation---that does not appear in the basin or on the ridge. The
susceptibility surface similarly shows that this entire region is highly
sensitive to noise: $\chi>0$ across most of the band, with a tangled
$\chi=0$ contour delineating small pockets of local flattening. This band is
therefore a structured ``transitional'' regime where echoes are present but
fragile, shaped by the interplay between coupling and noise.

\paragraph{(3) Ridge of instantaneous coherence without temporal persistence.}
Along the noise--free axis and extending slightly into finite $\sigma$, the
system forms a sharp ridge where the CHSH functional attains its maximal
values ($|S|\approx 2.8$) and the long--lag echo is initially strong. Yet this
ridge lies entirely above the universal memory threshold: even infinitesimal
noise---on the order of $\sigma=0.002$---collapses long--lag memory from
$\rho_S\approx 0.95$ to zero. The ridge therefore supports \emph{instantaneous}
geometric coherence but not \emph{persistent} temporal memory. In this regime
curvature is dominated by negative short-- and long--lag values with at most a
shallow mid--lag hump, and the susceptibility is strongly positive, indicating
that even slight perturbations produce large changes in the correlation
profile.

\paragraph{Summary.}
These three regimes---(i) a noise--dominated basin with trivial memory,
(ii) a structured intermediate band with mid--lag enhancement, and
(iii) a sharp ridge of instantaneous but nonpersistent coherence---together
establish that CHSH memory in classical oscillators is organized not as a
single monotonic decay but as a multi--layered temporal structure. This
three--regime map provides the framework for the global interpretation in
Section~\ref{sec:discussion} and motivates the broader topographic analysis of
observer fields developed in Paper~3.


%=============================================================================
% FIGURE
%=============================================================================

\begin{figure}[H]
\centering
\includegraphics[width=\linewidth]{figs/fig5_angle_field_panel}
\caption{
    Angle-resolved CHSH field across the $(K,\sigma)$ plane.
    For each grid point the measurement angles $(a,a',b,b')$ were numerically
    optimized to maximize the continuous CHSH functional, yielding an optimal
    value $S^*(K,\sigma)$ and corresponding echo at lag $\tau=50$.
    On the coherent ridge the optimal angles remain close to the canonical
    configuration; in the intermediate band they rotate rapidly, signaling
    geometric shear; deep in the noise basin no angle choice recovers $|S|>2$.
}
\label{fig:fig5_angle_field_panel}
\end{figure}



% Section 3.5 - Angle-Resolved Observer Field
% Paper 2 - Section 3.5: Angle-Resolved Geometry of the CHSH Field
% Camera-ready draft from Kelly

\subsection{Angle-Resolved Geometry of the CHSH Field}
\label{sec:angle_field}

The previous sections established that CHSH structure in the two-oscillator
system is not a single quantity but a multi-layered field with distinct
temporal components. Instantaneous correlations remain strong across the
high-$|S|$ ridge, whereas long-lag persistence collapses at a universal noise
threshold $\sigma_{\mathrm{mem}}\approx 0.002$. Curvature and susceptibility
maps revealed additional structure, including sign changes in the intermediate
band and a sharp coherent spine along the ridge. Together these results suggest
that the CHSH functional, treated as a directional probe, must itself vary
meaningfully with measurement geometry.

In this section we characterize the \emph{angle-resolved CHSH field} across the
$(K,\sigma)$ plane. For each grid point we scan over all measurement geometries
$(a,a',b,b')$ and identify both the optimal angles that maximize $|S|$ and the
local orientation of the CHSH gradient. This analysis transforms the CHSH
surface from a scalar field into a four-dimensional directional object whose
shape reflects how the system responds to perturbations in measurement
geometry.

The key advantage of this approach is that angle dependence reacts more
sensitively to noise than the absolute value of $|S|$. Near the coherent ridge
the optimal angles remain close to the canonical $(0^\circ,98^\circ,45^\circ,
127^\circ)$ configuration used in Paper~1. As noise increases the optimal
angles twist systematically, forming a smooth rotational flow that predicts the
location of the collapse boundary. Inside the intermediate band, where long-lag
memory has already failed but instantaneous coherence survives, the angle field
develops shear and local vortices indicative of competing phase preferences.
Deep inside the noise basin, the angle field degenerates, with the optimal
angles drifting freely and $|S|<2$ for all configurations.

This angle-resolved view completes the geometric decomposition of the CHSH
landscape. Instead of a single ``ridge,'' the system exhibits: (i) a coherent
spine where optimal geometry is stable; (ii) a twisting transition region where
noise induces rotation and shear in the angle field; and (iii) a noise basin in
which directional structure collapses entirely.

% Figure moved to results_3_4_chi.tex per patch instructions

Figure~\ref{fig:fig5_angle_field_panel} summarizes the angle-resolved CHSH field
obtained in Experiment E231. For each point in the $(K,\sigma)$ plane we
numerically optimized the measurement angles to maximize the continuous
CHSH functional, yielding an optimal value $S^*(K,\sigma)$ and a
corresponding echo value at lag $\tau=50$. Across the full grid the
optimized CHSH surface spans the range $1.45 \lesssim S^* \lesssim 2.83$
with a mean of $S^* \approx 2.71$: deep inside the noise basin even the
best measurement geometry cannot recover a violation ($S^*<2$), whereas on
the coherent ridge the optimized values remain close to the algebraic
maximum. At $\sigma=0$ the optimal-angle configurations preserve perfect
temporal memory with $\rho_S(\tau) = 1.0$, but away from this deterministic
limit the echo at the optimal geometry degrades, taking values as low as
$\rho_S \approx 0.43$.

The optimized angles themselves vary substantially across the field: the
best setting $a$ spans approximately $29^\circ$--$179^\circ$ and $b'$ spans
$13^\circ$--$186^\circ$, indicating strong twisting of the preferred
measurement geometry as $(K,\sigma)$ changes. On the coherent ridge the
optimal angles remain close to the canonical CHSH configuration, but in the
intermediate band they rotate rapidly, signaling geometric shear and
competing local optima in the CHSH field. These patterns reinforce the
three-regime picture: a well-aligned ridge with high $S^*$ and strong
memory, a geometrically unstable intermediate band with reduced echo, and a
noise basin in which CHSH structure cannot be recovered by any angle
choice.



%=============================================================================
\section{Discussion}
%=============================================================================

% Paper 3 - Discussion
% Scaffold - to be written

% PLACEHOLDER

The extension from two-oscillator to three-oscillator systems reveals that CHSH
memory is fundamentally shaped by network topology. Each motif---chain, star,
and triangle---exhibits distinct correlation patterns while sharing the
universal memory threshold $\sigma_{\mathrm{mem}}\approx 0.002$ established in
Paper~2.

% TODO: Interpret topology-dependent effects
% TODO: Connect to broader network theory
% TODO: Implications for larger systems
% TODO: Limitations and future directions



%=============================================================================
\section{Conclusion}
%=============================================================================

% Paper 2 - Conclusion
% Camera-ready draft from Kelly

This work extends the CHSH landscape introduced in Paper~1 by resolving its
temporal structure across coupling and noise. By measuring lag-resolved
correlations, curvature, echo strength, and susceptibility across a
high-resolution $(K,\sigma)$ grid, we find that CHSH memory in classical
two--oscillator dynamics is organized into three robust regimes: a
noise-dominated basin of trivial memory, a structured intermediate band with
mid-range enhancement, and a narrow ridge of strong instantaneous coherence but
vanishing temporal persistence.

The discovery of a universal memory threshold
$\sigma_{\mathrm{mem}} \approx 0.002$, far below the CHSH collapse curve
$\sigma_{\mathrm{c}}(K)$, highlights a sharp separation of scales between
instantaneous and lagged correlations. This separation imposes a concrete
limitation on the temporal coherence of CHSH-like dynamics: even systems that
exhibit strong instantaneous geometry cannot retain long-lag memory under
noise. The curvature and susceptibility analyses further show that the
intermediate regime contains nontrivial temporal structure that cannot be
inferred from the CHSH functional alone.

These findings suggest a new perspective for interpreting CHSH-like dynamics:
rather than a single coherent structure, the CHSH ridge is embedded within a
multi-layered temporal environment whose properties depend sensitively on both
lag and noise. This layered organization provides a principled vocabulary for
describing ``instantaneous coherence'' versus ``temporal coherence'' in
classical systems and sets the stage for extending the framework to larger
networks.

Paper~3 applies this temporal vocabulary to three-oscillator motifs---chain,
star, and triangle topologies---revealing how CHSH-like memory structures behave
when embedded in small networks with frustration, mediated coupling, and
cyclic flows. Together, the two papers establish a coherent path from
two-oscillator CHSH geometry to the topographic analysis of observer fields in
larger classical systems.



%=============================================================================
% References
%=============================================================================

\bibliographystyle{unsrt}
\bibliography{bib/references}


%=============================================================================
% Appendices (if needed)
%=============================================================================

% \appendix
% \input{sections/appendix}


\end{document}
