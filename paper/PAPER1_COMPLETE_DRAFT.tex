\documentclass[11pt,a4paper]{article}
\usepackage[utf8]{inputenc}
\usepackage{amsmath,amssymb,amsfonts}
\usepackage{graphicx}
\usepackage{hyperref}
\usepackage[margin=1in]{geometry}
\usepackage{mathptmx}
\usepackage{float}
\usepackage[font=small,labelfont=bf]{caption}
\usepackage{parskip}
\setlength{\parindent}{15pt}
\setlength{\parskip}{0pt}
\usepackage{physics}
\usepackage{titlesec}
\usepackage{tocloft}

% Equation spacing
\setlength{\abovedisplayskip}{10pt}
\setlength{\belowdisplayskip}{10pt}

% Section header formatting with normalsize enforcement
\titleformat{\section}{\normalsize\normalfont\bfseries\Large}{\thesection}{1em}{}[\normalsize]
\titleformat{\subsection}{\normalsize\normalfont\bfseries\large}{\thesubsection}{1em}{}[\normalsize]
\titleformat{\subsubsection}{\normalsize\normalfont\bfseries}{\thesubsubsection}{1em}{}[\normalsize]

% TOC formatting
\cftsetpnumwidth{2em}
\cftsetrmarg{3em}

\title{Continuous-Angle CHSH Correlations in Noisy Coupled Oscillators:\\
A Systematic Parameter-Space Study\\[10pt]
\large Paper 1: The CHSH Landscape}
\author{Kelly McRae}
\date{November 2025}

\hypersetup{
    colorlinks=true,
    linkcolor=blue,
    citecolor=blue,
    urlcolor=blue
}

\begin{document}
\normalfont
\maketitle

\begin{abstract}
We demonstrate that the CHSH correlation functional---traditionally applied to test local hidden-variable theories---provides a quantitative diagnostic for phase-space geometry in classical nonlinear dynamics. Applying $S = E(a,b) - E(a,b') + E(a',b) + E(a',b')$ to a noisy Kuramoto-type system of coupled phase oscillators, we find maximum correlations of $|S| = 2.819 \pm 0.003$, approaching the algebraic maximum of $2\sqrt{2} \approx 2.828$ for the CHSH functional. To our knowledge, this is the first systematic mapping of a continuous-phase CHSH functional exceeding $|S| \approx 2.8$ in a deterministic two-oscillator model.

We systematically map correlation structure across parameter space, revealing two novel features: (i) a linear scaling law for noise-induced correlation collapse, $\sigma_c(K) = 0.602K + 0.222$ (R² = 0.984), and (ii) persistent temporal coherence ($\rho_S \approx 0.86$) in regimes where instantaneous correlations have returned to classical bounds. These features are not captured by traditional synchronization metrics and indicate that CHSH functionals detect correlation memory beyond phase-locking. Additionally, we observe strong parameter specificity---randomly chosen configurations yield $|S| \approx 1.0$, with only 7\% exceeding 2.0---confirming that high correlations require deliberate optimization rather than generic coupling.

Because our system exhibits explicit coupling and continuous phase dynamics, the bound $|S| \leq 2$---which constrains local hidden-variable theories---does not apply. Our results establish that $|S| > 2$ is a signature of correlation structure achievable through multiple physical mechanisms and does not require quantum entanglement or violation of locality. CHSH-type functionals thus provide precise diagnostic tools for nonlinear dynamical systems.
\end{abstract}

\newpage
\tableofcontents
\newpage


\section{Introduction}

\subsection{Motivation}

Bell inequalities, and in particular the CHSH form

\[S = E(a,b) - E(a,b') + E(a',b) + E(a',b'),\]

play a central role in distinguishing classical local models from quantum correlations. In quantum theory, exceeding the CHSH bound $|S| \leq 2$ arises from entangled two-qubit states and dichotomic measurements, with an algebraic maximum of $2\sqrt{2}$ (often called the Tsirelson bound in the quantum information literature).

Outside of quantum mechanics, many dynamical systems generate non-trivial correlation structures, but their capacity to produce $|S| > 2$ --- and the precise conditions under which such high correlations appear --- remain poorly understood. Nonlinear phase-coupled oscillators, such as variants of the Kuramoto model, exhibit robust synchronization, persistent phase relations, and rich temporal structure. Yet their CHSH landscape has not been systematically explored.

This work performs a controlled, high-resolution study of CHSH correlations generated by a pair of coupled nonlinear oscillators evolving under noise, detuning, and tunable measurement geometry. The goal is not to model quantum systems, but to characterize the conditions under which a classical continuous dynamical system can produce CHSH values above the classical bound, and to identify the structural features that enable or suppress such high correlations.


\subsection{Background and Context}

Two-oscillator Kuramoto-type models provide a minimal setting where:

\begin{itemize}
  \item phase coherence (synchronization strength)
  \item frequency mismatch (detuning)
  \item coupling strength
  \item external noise
  \item measurement angle geometry
\end{itemize}

jointly determine the instantaneous correlation structure of the phase difference $\Delta\theta(t)$.

Earlier exploratory simulations in our laboratory (unpublished) showed that this model can produce CHSH values in the range $2 < |S| < 2.8$. These trials revealed the existence of a high-correlation regime but did not establish its structure or boundaries.

Here we present the first systematic characterization of that regime across noise, detuning, coupling strength, and measurement geometry.


\subsubsection{Related Work}

Bell's theorem and the CHSH inequality were originally formulated in the context of quantum spin and polarization measurements, where dichotomic outcomes and photon counting are natural. Over the last two decades, however, several authors have emphasized that Bell-type correlations can appear in systems that are "classical" in other senses. Spreeuw and others introduced the idea of \textit{classical entanglement} in multimode optical fields, showing that nonseparable correlations between polarization and spatial modes of a single beam can mimic quantum entanglement when described in a Hilbert-space framework [1]. Subsequent work on classical coherence and "entangled" classical light has demonstrated violations of Bell-like inequalities using carefully prepared optical fields and coherence functions, without invoking single-photon states [2,3].

A complementary line of work asks which mathematical ingredients of quantum theory are responsible for Bell violations. De Zela and others have constructed local realist models based on classical optics and inner-product probability measures that reproduce quantum-like correlations, arguing that Hilbert-space geometry, rather than nonlocality per se, underlies Bell's theorem [4]. There are also semiclassical models in which classical fields combined with quantum detection theory can violate Bell inequalities [5]. In a different direction, Gerhardt \textit{et al.} showed that purely classical light can produce apparent violations when detector-control loopholes are exploited, underscoring the importance of strict experimental assumptions [6].

The present work differs from these approaches in three ways. First, we study a deterministic dynamical system---a pair of locally coupled phase oscillators with tunable coupling strength, frequency mismatch, and additive noise---rather than static optical fields or abstract probabilistic models. Second, we compute the standard CHSH functional directly from the continuous phase trajectories, without photon counting, threshold detection, or post-selection. Third, we map an explicit phase diagram in the space of coupling and noise, identifying a linear collapse boundary ($\sigma_c(K)$), an optimal measurement geometry, and a frequency-mismatch "sweet spot" where $|S|$ nearly saturates the Tsirelson bound. To our knowledge, no previous work has reported $|S| > 2$ emerging from such a noisy deterministic phase-locking mechanism, nor documented the persistence of temporal memory after the CHSH value has returned to the $|S| < 2$ regime.


\subsection{Objectives of This Paper}

The present work aims to answer four foundational questions:

\subsubsection{Under what conditions do nonlinear phase-coupled oscillators produce CHSH values $|S| > 2$?}

We analyze how these correlations depend on:
\begin{itemize}
  \item coupling strength K,
  \item frequency mismatch $\Delta\omega$,
  \item measurement geometry (angle differences),
  \item and additive noise $\sigma$.
\end{itemize}

\subsubsection{How robust are these high correlations to noise?}

We measure the critical noise strength $\sigma_c$ at which $|S|$ drops below 2 for given values of K. A dense sweep of coupling strengths reveals a clean linear scaling law:

\[\sigma_c(K) = 0.602 K + 0.222,\]

with R² = 0.984, indicating strong coupling-dependence and a non-zero noise intercept whose origin remains under investigation.

\subsubsection{Which measurement geometries maximize CHSH values?}

A full angle sweep identifies:
\begin{itemize}
  \item a broad ridge of high $|S|$ configurations,
  \item a robust optimum near ($\Delta a$, $\Delta b$) $\approx$ (98°, 82°),
  \item and $|S|$ values approaching 2.81, close to the Tsirelson limit of $2\sqrt{2} \approx 2.828$.
\end{itemize}

\subsubsection{How do detuning and synchronization quality interact with CHSH structure?}

We discover that:
\begin{itemize}
  \item moderate frequency mismatch ($\Delta\omega \approx 0.1$) enhances $|S|$ relative to perfect matching,
  \item perfect frequency matching ($\Delta\omega = 0$) yields slightly lower $|S|$ due to rigid locking,
  \item excessive mismatch ($\Delta\omega \gtrsim 0.2$) weakens correlation amplitude.
\end{itemize}

This reveals a well-defined optimal detuning at the ridge point, with broader $\Delta\omega$--$K$ landscape exploration reserved for future work.


\subsection{Scope of This Paper}

This paper is intentionally narrow in scope. We do not address loophole-free Bell tests, quantum interpretations, or claims about quantum foundations. Our system explicitly violates the locality assumption of Bell's theorem (via coupling term K), and thus $|S| > 2$ does not contradict any established physics. Our aims are strictly:

\begin{itemize}
  \item to map the CHSH landscape generated by a classical continuous dynamical system,
  \item to identify the structural ingredients required for $|S| > 2$,
  \item to quantify robustness using large controlled sweeps, and
  \item to establish a reproducible experimental suite for future exploration.
\end{itemize}

All simulations reported here use continuous measurement outcomes (not dichotomic signs), and all results are reproducible via scripts included with this paper.


\subsection{Contributions}

This work provides:

1. the first detailed $\sigma_c(K)$ scaling law for CHSH collapse in a continuous oscillator model;
2. the first high-resolution angle-space map showing the ridge and optimal geometry;
3. the first systematic $\Delta\omega$ sweep identifying an "optimal tension" regime maximizing $|S|$;
4. a minimal echo panel demonstrating that temporal memory persists in regimes where $|S|$ has returned to values below 2;
5. a complete and reproducible experimental suite (A1--A5, B1).

Together, these results establish a coherent and internally consistent picture of when and how nonlinear oscillators can exceed the classical CHSH bound.


\subsection{Outline of the Paper}

\textbf{Section II} describes the model and methodology.

\textbf{Section III} presents the main results:
\begin{itemize}
  \item \textbf{A1}: Noise-robustness and $\sigma_c$ scaling
  \item \textbf{A2}: Angle-space ridge and optimal geometry
  \item \textbf{A3}: Detuning sweeps and optimal $\Delta\omega$
  \item \textbf{B1}: Minimal echo panel
\end{itemize}

\textbf{Section IV} discusses implications and limitations.

\textbf{Section V} concludes with open questions and directions for follow-up studies.
\section{Model and Methods}

\subsection{Dynamical Model}

We study a pair of coupled phase oscillators evolving according to a noisy Kuramoto-type dynamic:

\[\dot{\theta}_A = \omega_A + K\sin(\theta_B - \theta_A) + \eta_A(t),\]
\[\dot{\theta}_B = \omega_B + K\sin(\theta_A - \theta_B) + \eta_B(t).\]

Here:

\begin{itemize}
  \item $\theta_A(t)$, $\theta_B(t) \in [0, 2\pi)$ are oscillator phases
  \item $\omega_A$, $\omega_B$ are intrinsic frequencies
  \item $\Delta\omega = \omega_B - \omega_A$ is the detuning (frequency mismatch)
  \item K is the symmetric coupling strength
  \item $\eta_{A,B}(t)$ are independent Gaussian noise terms with $\eta(t) \sim \mathcal{N}(0, \sigma^2)$.
\end{itemize}

The model exhibits expected features of nonlinear synchronization: deterministic drift toward phase locking for $K \gtrsim \Delta\omega$, degradation of synchrony with increasing noise $\sigma$, and full desynchronization in the high-noise regime.


\subsection{Numerical Integration}

All simulations use explicit Euler integration:
\[
\theta(t+\Delta t) = \theta(t) + \Delta t \cdot \dot{\theta}(t),
\]
with:

\begin{itemize}
  \item \textbf{time step}: $\Delta t = 0.01$
  \item \textbf{total time}: $T = 1000$ (100,000 integration steps)
  \item \textbf{transient discarded}: first 20 time units
  \item \textbf{phases wrapped} modulo $2\pi$ after every update
\end{itemize}

For each experimental condition, we generate 10 independent trajectories using different random seeds to estimate mean and variance. All error bars denote standard error of the mean (SEM) across independent seeds. Distributions are approximately Gaussian across all tested regimes, justifying SEM as an appropriate uncertainty measure.

All code and configuration files are included in the supplementary repository. Full reproducibility is provided via:

\begin{verbatim}
bash RUN_ALL_PAPER1.sh
\end{verbatim}

\textbf{Numerical Convergence:} To verify precision of the reported Bell values (maximum $|S| = 2.819 \pm 0.003$), we performed a convergence test by reducing the integration step from $\Delta t = 0.01$ to $\Delta t = 0.001$ for $N = 10$ representative trajectories. The resulting variation $\Delta|S| < 0.001$ confirms that numerical error is well below measurement uncertainty.

All reported uncertainties are standard errors of the mean (SEM) across independent random seeds, reflecting seed-to-seed variability in stochastic trajectory evolution. Systematic errors from finite integration time are estimated as $< 0.001$ in $|S|$ based on convergence tests extending integration to $20,000\tau$ (not shown); interpolation uncertainty in $\sigma_c$ estimates is $< 0.005$ based on quadratic fit residuals.


\subsection{Continuous-Variable CHSH Functional}

To quantify correlation strength between the two oscillators, we use a continuous-variable (CV) analogue of the CHSH functional constructed from cosine-valued correlators. For any choice of measurement angles a, b, c, d, we define

\[E(a,b) = \left\langle \cos\!\left[(\theta_A + a)-(\theta_B + b)\right]\right\rangle,\]

where the average is taken over the stationary portion of each trajectory. This choice is natural for phase-coupled oscillators: unlike discrete $\pm 1$ dichotomizations, the cosine correlator preserves the full rotational symmetry of the underlying system and avoids threshold-dependent artifacts. Similar CV-type formulations appear in continuous-angle Bell tests and in phase-correlation studies of coupled rotors.

The CHSH functional is then constructed in the usual form

\[S(a,b,c,d) = E(a,b) + E(a,d) + E(c,b) - E(c,d).\]

For deterministic synchronized phases, $|S|$ attains its algebraic maximum of 2. Noise broadening decreases the magnitude of each correlator, reducing $|S|$ accordingly. This framework therefore provides a sensitive probe of how coupling, noise strength, and detuning shape the cross-oscillator phase structure.

The continuous formulation used here differs from the discrete CHSH test only in the choice of observable; it retains the same combination of four correlators and the same classical upper bound $|S| \leq 2$. Throughout the paper, we avoid interpreting $|S| > 2$ as a nonclassical violation; instead, we view it as a quantitative descriptor of structured phase correlations in a noisy dynamical system. This interpretation is consistent with the model's classical nature and with standard usage of CV-CHSH functionals in non-quantum contexts.

Continuous-outcome CHSH functionals of this form have been widely used in both quantum and classical phase-space analyses. In the quantum optics literature, Banaszek and Wódkiewicz established continuous-variable Bell tests based on displaced-parity correlators and phase-space sampling, later extended through continuous-quadrature formulations by Gilchrist et al., Ralph et al., and Wenger et al. Classical and semiclassical analogs have also employed continuous-angle or continuous-quadrature correlators to probe field coherence and nonlocal-like structure in wave systems. Our use of $E(a,b) = \langle\cos(\Delta\theta + \Delta\phi)\rangle$ falls within this established class of continuous-variable CHSH functionals, where $|S| > 2$ signifies a structural departure from the classical CHSH bound for dichotomic outcomes but does not imply quantum nonlocality.

Unless otherwise specified, we use the optimal angles identified in Experiment A2:

\begin{itemize}
  \item a = 0°
  \item a' = 95°
  \item b = 45°
  \item b' = 129°
\end{itemize}

which maximize $|S|$ for this dynamical system.


\subsection{Parameter Sweeps and Experimental Conditions}

Across the four experiments of Paper 1, we systematically vary:

\subsubsection{Coupling strength K}

$K \in \{0.3, 0.5, 0.7, 0.9\}$

Used in A1 (scaling law) and B1 (memory panel).

\subsubsection{Noise amplitude $\sigma$}

$\sigma \in \{0, 0.1, 0.2, 0.3, 0.4, 0.5, 0.7, 0.9, 1.0\}$

Used in A1 (collapse curves) and B1.

\subsubsection{Frequency mismatch $\Delta\omega$}
$\Delta\omega \in \{0.1, 0.2, 0.3, 0.4, 0.5\}$

Used in A3 (detuning sweep).

\subsubsection{Measurement geometry}
$(a, a', b, b') \in [0^\circ, 180^\circ]$

Sampled on a $181 \times 181$ grid in A2 (global angle ridge).


\subsection{Derived Metrics}

In addition to the CHSH parameter S, we compute:

\subsubsection{Phase Coherence (Order Parameter)}

The Kuramoto order parameter quantifies phase synchronization strength:

\[r = \left|\left\langle e^{i(\theta_A(t) - \theta_B(t))}\right\rangle_t\right|\]

where the time average is taken over the post-transient time segment. This metric takes values in [0,1], with r = 0 indicating uniformly distributed phase differences between oscillators and r = 1 indicating perfect phase locking. This follows standard order-parameter constructions in coupled-oscillator theory (Pikovsky et al., 2001; Acebrón et al., 2005).

\textit{Note.} This differs from the EEG "Phase Lag Index" of Stam et al. (2007), which was designed to suppress zero-lag synchronization artifacts in neural connectivity analysis. Our metric directly measures phase coherence without such filtering.

\subsubsection{Autocorrelation of CHSH Time Series $\rho_S$($\tau$)}

For each trajectory, we compute the CHSH instantaneous value:

\begin{align*}
S_{\text{inst}}(t) &= \cos(\theta_A + a - \theta_B - b) \\
                   &\quad - \cos(\theta_A + a - \theta_B - b') \\
                   &\quad + \cos(\theta_A + a' - \theta_B - b) \\
                   &\quad + \cos(\theta_A + a' - \theta_B - b').
\end{align*}

Then compute:

\[\rho_S(\tau) = \text{corr}(S_{\text{inst}}(t), S_{\text{inst}}(t+\tau)),\]

with lag $\tau = 10$ steps.

This provides a measure of temporal coherence of the CHSH observable.

\subsubsection{Correlation Amplitude}

The time-averaged CHSH amplitude $\langle |S| \rangle$ is used to track approach to the classical bound $|S| = 2$.


\subsection{Definition of the Collapse Threshold}

Throughout the manuscript we define the noise threshold $\sigma_c(K)$ as the
value of $\sigma$ at which the CHSH amplitude crosses $|S| = 2.3$. This choice
requires explanation. In principle the natural classical boundary is $|S| = 2$,
but in practice the deterministic two-oscillator system exhibits a shallow,
extended shoulder just above this value due to finite-time averaging and
trajectory-to-trajectory fluctuations. Over the range $2.0 < |S| < 2.3$ the
system does not yet enter the fully classical regime: the coherence remains
partially intact, the phase distribution retains non-Gaussian structure, and
the effective drift-to-diffusion ratio is still above the long-term classical
limit.

Empirically, the derivative $\partial|S|/\partial\sigma$ becomes steep and
monotonic only once $|S|$ falls below approximately $2.3$. In this regime the
CHSH amplitude becomes insensitive to seed variability and no longer exhibits
noise-induced plateaus or locking artifacts. Using $2.3$ as the threshold
therefore provides a robust and reproducible marker of the onset of true
correlation collapse, whereas using the formal bound $|S| = 2$ tends to obscure
the transition due to the shallow approach and elevated variance in the
shoulder region.

To quantify the collapse transition, we computed the derivative $\partial|S|/\partial\sigma$ across the noise range for $K=0.7$ (Fig. S1). The derivative exhibits a clear inflection: at $\sigma \approx 0.15$--$0.20$, where $|S| \approx 2.0$--$2.3$, the slope $\partial|S|/\partial\sigma \approx -1.8$ to $-2.1$, while at $\sigma \approx 0.80$--$1.0$, the slope softens to $\partial|S|/\partial\sigma \approx -0.7$ to $-0.9$. This three-fold steepness reduction motivates our choice of $|S| = 2.3$ as a practical threshold capturing the onset of rapid collapse rather than merely the mathematical crossing of the classical bound. As $\sigma$ increases from zero, the CHSH value remains stable until the noise begins to compete directly with the coupling-driven phase alignment. Near $|S| \approx 2.0$--$2.3$, a small increase in $\sigma$ rapidly disrupts the locked phase relation, producing a steep $\partial|S|/\partial\sigma$. This marks the onset of the transition from a partially coherent state to a noise-dominated regime where coupling and diffusion have comparable strength. Table S1 compares extracted $\sigma_c$ values for both $|S| = 2.0$ and $|S| = 2.3$ thresholds; the linear scaling $\sigma_c = \alpha K + \beta$ holds for both definitions with comparable fit quality ($R^2 > 0.98$), differing only in the intercept $\beta$ by approximately $0.156$. We use the $2.3$ threshold throughout to minimize sensitivity to finite-time averaging artifacts near the classical boundary.

We emphasize that this is a \emph{practical} threshold rather than a theoretical
boundary: the classical limit remains $|S| \le 2$, but the value $2.3$ yields a
clearer, more stable operational definition of the point at which the system's
high-correlation regime can no longer be sustained.


\subsection{Reproducibility}

All runners for Paper 1 are provided in the repository under:

\begin{verbatim}
analysis/scripts/paper1_runners/
\end{verbatim}

The entire study --- 1,680 trajectories --- can be reproduced with:

\begin{verbatim}
bash RUN_ALL_PAPER1.sh
\end{verbatim}

which executes A1 → A2 → A3 → B1 sequentially and produces all figures used in the paper.

\textbf{Computational Environment:} All simulations performed using Python 3.9+ with NumPy 1.21+ and Matplotlib 3.4+ for visualization. Total runtime for the complete study is approximately 45 minutes on a standard laptop (single-core execution). Full dependency specifications and environment configuration provided in the repository.

Random seeds for each trajectory are set using a standard reproducible generator and logged for all runs.


\subsection{Notes}

\begin{itemize}
  \item All numerical values reported in Results use post-transient statistics
  \item Error bars represent standard error across independent seeds
  \item Critical noise $\sigma_c$ is identified by interpolation where $\langle|S|\rangle$ crosses 2.3
  \item Angle optimization (A2) uses fine 1° grid near expected optimum
\end{itemize}
\section{Results}


\begin{figure}[H]
\centering
\includegraphics[width=0.9\textwidth]{figures/fig1_combined.png}
\caption{
CHSH Correlation Landscape in Parameter Space.
Heatmap showing CHSH amplitude $|S|$ as a function of coupling strength $K$ and noise amplitude $\sigma$.
The high-correlation region ($|S| \gtrsim 2.3$, yellow/green) is bounded by the collapse threshold
$\sigma_c(K) \approx 0.60K + 0.22$ (black dashed line, fitted to $K \in [0.3,0.9]$).
White circles mark tested parameter combinations at $\sigma = 0.2$.
The threshold region $0.1 < K_{\min} < 0.2$ (where high correlations first become accessible)
is visible at low $K$ where the boundary line begins.
The optimal point ($K = 0.7$, $\sigma = 0.2$, gold star) yields $|S|_{\max} = 2.819$.
Inset: 3D surface visualization showing the topographic structure of the correlation landscape.
The surface is constructed from 144 measurements across the tested parameter space (Experiment A1).
}
\end{figure}
\normalsize


\subsection{Noise-Induced Collapse of High CHSH Correlations (Experiment A1)}

We first characterize how high CHSH correlations degrade as external noise is increased. For each coupling strength $K \in \{0.3, 0.5, 0.7, 0.9\}$, we sweep the noise amplitude $\sigma \in [0, 1]$ and measure the correlation amplitude $\langle|S|\rangle$ over ten independent trajectories.

\subsubsection{High-Correlation Regime and Collapse Point}

For all coupling strengths tested, the system exhibits a robust $|S| > 2$ region at low noise, with maximum values:

\[|S|_{\max} \approx 2.815 \pm 0.005,\]

close to the Tsirelson bound of 2.828. As noise increases, correlations degrade smoothly until a sharp collapse point is reached where $\langle|S|\rangle$ falls below the classical boundary $|S| = 2$.

We define the collapse threshold $\sigma_c(K)$ as the noise amplitude where $\langle|S|\rangle$ crosses 2.3, identified by linear interpolation between adjacent grid points. This threshold lies midway between the classical Bell value (2.0) and maximal correlations ($\sim$2.8), providing a robust marker insensitive to statistical fluctuations while clearly indicating exit from the high-correlation regime.

The empirically determined thresholds are:

\begin{align*}
K = 0.3 &\Rightarrow \sigma_c = 0.384 \\
K = 0.5 &\Rightarrow \sigma_c = 0.545 \\
K = 0.7 &\Rightarrow \sigma_c = 0.654 \\
K = 0.9 &\Rightarrow \sigma_c = 0.749
\end{align*}

Across all couplings, the collapse region is narrow and well-defined, indicating that the CHSH observable is sensitive to dynamical decoherence in a controlled and repeatable way.

\subsubsection{Scaling Law and Saturation of Noise Robustness}

To characterize how noise robustness depends on coupling strength, we performed an extended sweep over $K \in \{0.1, 0.2, 0.3, 0.4, 0.5, 0.6, 0.7, 0.8, 0.9, 1.0, 1.2, 1.5\}$. For each $K$ we determined the critical noise $\sigma_c$ at which $\langle|S|\rangle$ falls below 2.0. The results are:

\begin{align*}
K = 0.1 &\Rightarrow \sigma_c \text{ not observed (no } |S|>2 \text{ in tested range)} \\
K = 0.2 &\Rightarrow \sigma_c = 0.235 \\
K = 0.3 &\Rightarrow \sigma_c = 0.384 \\
K = 0.4 &\Rightarrow \sigma_c = 0.476 \\
K = 0.5 &\Rightarrow \sigma_c = 0.545 \\
K = 0.6 &\Rightarrow \sigma_c = 0.600 \\
K = 0.7 &\Rightarrow \sigma_c = 0.654 \\
K = 0.8 &\Rightarrow \sigma_c = 0.709 \\
K = 0.9 &\Rightarrow \sigma_c = 0.749 \\
K = 1.0 &\Rightarrow \sigma_c = 0.787 \\
K = 1.2 &\Rightarrow \sigma_c = 0.865 \\
K = 1.5 &\Rightarrow \sigma_c = 0.970
\end{align*}

Over the mid-range $K \in [0.3, 0.9]$, these points fall on an almost perfect straight line. A least-squares fit in this interval yields an empirical scaling law

\[\sigma_c(K) \approx 0.60K + 0.22 \quad (R^2 \approx 0.98)\]

in excellent agreement with the simpler four-point fit reported in the initial A1 runs. Within this window, stronger coupling increases resistance to noise in a nearly linear fashion, consistent with a simple competition between coupling-driven phase locking and noise-driven diffusion.

The extended sweep, however, shows that this linearity is local, not global. At very weak coupling (K = 0.1), the system fails to produce $|S| > 2$ for any $\sigma$ in the tested range, implying that $\sigma_c$ lies below our resolution or that no high-correlation regime exists at all for such small K. This immediately rules out a true nonzero intercept at K → 0: the line $\sigma_c = 0.60K + 0.22$ should be understood as a tangent approximation over intermediate K, not a fundamental law extrapolating to vanishing coupling.

At strong coupling (K $\gtrsim$ 1.0), $\sigma_c(K)$ continues to increase but with a noticeably reduced slope. Between K = 0.9 and K = 1.5 the effective slope drops to $\approx$0.37, and the curve visibly bends away from the mid-range linear trend. This high-K shoulder reflects a saturation of noise robustness: once the sin($\Delta\theta$) coupling has driven the oscillators into a tightly locked regime, further increases in K provide diminishing returns against large noise. Together, the low-K failure to reach $|S|$ > 2 and the high-K saturation establish $\sigma_c(K)$ as a monotone, concave-down curve with a broad linear regime in the middle rather than a globally linear scaling. This curvature is consistent with the breakdown of the linearized drift approximation at weak coupling and with nonlinear saturation of the restoring force at strong coupling (Sec. 4.8.1).

Although the global $\sigma_c(K)$ curve is clearly nonlinear---showing low-K suppression and high-K saturation---the mid-range interval $K \in [0.3, 0.9]$ exhibits an effectively constant slope. Over this domain the data are well-fit by the empirical tangent approximation $\sigma_c \approx 0.60K + 0.22$ (R² $\approx$ 0.98), which we use only as a compact summary of the local trend.

An extended analysis of $\sigma_c(K)$ across the full coupling range $K \in [0.1, 2.5]$ (Fig. S3, Table S3) confirms three distinct regimes. For $K < 0.2$, no violations with $|S| \ge 2.3$ are observed within the tested noise range $\sigma \in [0, 2.5]$, indicating insufficient coupling to sustain correlated dynamics. At low coupling the phase tension between oscillators is too weak to maintain meaningful alignment; the sine interaction has little leverage, so drift driven by noise dominates the dynamics, keeping the system in the strictly classical regime. For $K \in [0.3, 0.9]$, $\sigma_c(K)$ follows the linear scaling $\sigma_c = 0.596K + 0.231$ (R² = 0.985), consistent with the mid-range behavior analyzed above. For $K > 1.5$, $\sigma_c(K)$ exhibits upward curvature and departs from the linear trend, reflecting increased noise robustness in the strong-coupling regime. With large $K$, the oscillators lock so tightly that relative phase fluctuations are strongly suppressed; the phase difference $\Delta\theta$ becomes almost rigid, and additional noise has diminished effect. This saturation demonstrates that the linear scaling is a mid-range tangent approximation rather than a fundamental law extending to arbitrary coupling strengths.

The approximate linear regime $\sigma_c(K) \approx 0.60K + 0.22$ and its deviations at low and high K arise naturally from a drift--diffusion balance in the phase-difference dynamics. A detailed discussion is provided in Sec. 4.8.1.


\begin{figure}[H]
\centering
\includegraphics[width=0.9\textwidth]{figures/fig2_sigma_c_scaling.png}
\caption{Noise-Coupling Scaling Law}
\end{figure}
\normalsize

Figure 2: Global $\sigma_c(K)$ curve with mid-range linear regime. Critical noise amplitude $\sigma_c$ at which $|S|$ drops below 2 as a function of coupling strength K. Circles show empirical values obtained from the extended sweep $K \in \{0.1, 0.2, \ldots, 1.5\}$. The solid line shows the best-fit linear scaling $\sigma_c(K) \approx 0.60K + 0.22$ obtained by fitting only the mid-range points $K \in [0.3, 0.9]$. In this interval, $\sigma_c$ grows almost linearly with K (R² $\approx$ 0.98), indicating a simple balance between coupling-driven alignment and noise-driven diffusion. At very weak coupling (K = 0.1) no $|S|$ > 2 region is observed within the tested $\sigma$ range, and at strong coupling (K $\gtrsim$ 1.0) the curve bends away from the extrapolated line and begins to saturate. This establishes $\sigma_c(K)$ as a monotone, concave-down curve with a broad linear regime rather than a globally linear law.

\subsubsection{Collapse Curves and Universal Shape}

Figure 3 displays $\langle|S|\rangle$ as a function of $\sigma$ for all four coupling strengths. All curves share a common qualitative structure:

\begin{itemize}
  \item High-correlation plateau for $\sigma \lesssim \sigma_c/2$
  \item Smooth degradation as noise increases
  \item Sharp drop at the collapse threshold
  \item Classical saturation at $\langle|S|\rangle \approx 1.6$--1.8 for $\sigma \gtrsim 1.0$
\end{itemize}

The similarity of curve shapes across coupling strengths indicates a common collapse structure in this system, governed primarily by the competition between coupling-driven phase alignment and noise-driven diffusion. Whether this structure generalizes beyond the Kuramoto model would require comparison with other coupled-oscillator systems.


\begin{figure}[H]
\centering
\includegraphics[width=0.9\textwidth]{figures/fig3_S_vs_sigma.png}
\caption{Universal Collapse Curves for CHSH Correlations}
\end{figure}
\normalsize

Figure 3: Universal Collapse Curves for CHSH Correlations. Mean CHSH amplitude $\langle|S|\rangle$ as a function of noise strength $\sigma$ for four coupling values: K = 0.3 (purple), K = 0.5 (blue), K = 0.7 (green), K = 0.9 (orange). All curves exhibit: (i) high-correlation plateaus near the Tsirelson bound for low noise, (ii) smooth degradation through the transition zone, (iii) sharp collapse at the critical threshold $\sigma_c(K)$ (marked by vertical dashed lines), and (iv) classical saturation at $|S| \approx 1.6$-1.8 for high noise. Error bars show SEM across N = 10 independent trajectories. Horizontal dashed line marks the classical bound $|S| = 2$.

\subsubsection{Synchronization and CHSH Correlation Relationship}

Alongside the CHSH observable, we measure the phase coherence (Kuramoto order parameter):

\[r = \left|\left\langle e^{i(\theta_A - \theta_B)}\right\rangle\right|\]

(Here and throughout, $r$ denotes the Kuramoto order parameter measuring phase coherence, not the EEG Phase Lag Index.)

The order parameter $r$ decreases more gradually with noise than CHSH correlations. Even for $\sigma$ where $|S| < 2$, we typically observe:

\[r \approx 0.6 - 0.8,\]

indicating partial synchrony persists after $|S|$ has dropped below 2.

This confirms that:

\begin{itemize}
  \item Return to $|S|$ < 2 is not identical to loss of phase coherence,
  \item The CHSH observable is more sensitive to decoherence than the Kuramoto order parameter r, and
  \item $|S|$ dropping below 2 marks a stricter criterion than simple synchrony breakdown.
\end{itemize}

This distinction becomes important later when comparing memory and CHSH correlation structure.

\subsubsection{Summary of Experiment A1}

Experiment A1 yields three principal findings:

1. Clean high CHSH values up to $|S|$ $\approx$ 2.815 across all coupling strengths
2. A linear scaling law for noise robustness
3. A decoupling between simple synchrony (r) and CHSH correlation structure ($|S|$)

These results establish the foundational dynamical landscape on which the remaining experiments build.

\subsubsection{Parameter Specificity: Control Comparison (Experiment C1)}

To verify that high-$|S|$ correlations require specific parameter choices rather
than arising generically from coupled-oscillator configurations, we compared our
optimized parameters ($K = 0.7$, $\Delta \omega = 0.2$, $\sigma = 0.2$,
angles $= 0^\circ/95^\circ/45^\circ/129^\circ$) against $N = 100$ randomly
sampled configurations drawn uniformly from:
\[
K \in [0.3, 1.0],\quad
\sigma \in [0.1, 1.0],\quad
\Delta \omega \in [0.1, 0.5],\quad
\text{angles} \in [0^\circ, 180^\circ].
\]

Results: Random configurations yielded
$|S| = 1.061 \pm 0.659$ (only 7\% exceeding 2.0), while the optimized
configuration produced $|S| = 2.778 \pm 0.002$ (Cohen's $d = 3.68$, indicating a
very large effect size). The optimized parameters fall at the 99th percentile of
the random distribution (see Supplementary Material, Figure~S2).

Interpretation: In this Kuramoto-like system, high correlations emerge only
under carefully tuned parameter combinations rather than generically.


\subsection{Angle Optimization and Ridge Structure (Experiment A2)}

The CHSH parameter depends on the choice of measurement angles $(a, a', b, b')$. To identify the optimal geometry for this dynamical system, we perform a systematic scan over the angle separations:

\begin{align*}
\Delta\alpha &= a' - a  \in [80°, 110°] \\
\Delta\beta &= b' - b  \in [70°, 100°]
\end{align*}

with $a$ fixed at 0° and $b$ at 45°.

\subsubsection{Optimal Measurement Geometry}

Within the scanned parameter window ($\Delta\alpha \in [80°, 110°]$, $\Delta\beta \in [70°, 100°]$, with $a = 0°$ and $b = 45°$ fixed), the maximum is found at:

\[\Delta\alpha^* = 95°, \quad \Delta\beta^* = 84°\]

yielding:

\[|S|_{\max} = 2.819 \pm 0.003.\]

This corresponds to the explicit angle configuration:

\[a = 0°, \quad a' = 95°, \quad b = 45°, \quad b' = 129°,\]

which we adopt for all subsequent experiments.

The measured value $|S| = 2.819$ approaches---but remains slightly below---the Tsirelson bound of 2.828, suggesting that the continuous-phase measurement model introduces a small but systematic reduction compared to ideal quantum projections.

\subsubsection{Angles and Symmetry Considerations}

Because the correlation functional

\[E(a,b) = \langle \cos[(\theta_A + a) - (\theta_B + b)] \rangle\]

depends only on relative angle differences, global angle offsets produce no change in the measured correlations. Without loss of generality, we therefore fix one angle on Alice's side to a = 0° and one on Bob's side to b = 45°, and perform optimization over the remaining angle separations $\Delta\alpha$ = a' - a and $\Delta\beta$ = b' - b.

This reduces the four-angle parameter space to the two physically relevant degrees of freedom while preserving the full correlation structure of the CHSH functional in this model. The choice of b = 45° (rather than 0°) follows conventional Bell-test geometry where measurement axes are offset to probe correlation asymmetries.

Due to the $2\pi$ periodicity and cos() symmetry of the measurement functional, the scanned ranges $\Delta\alpha \in [80°, 110°]$ and $\Delta\beta \in [70°, 100°]$ capture the essential correlation landscape without requiring exhaustive 4D exploration.

\subsubsection{Broad Ridge Structure}

Figure 4 shows the two-dimensional landscape $|S|(\Delta\alpha, \Delta\beta)$ as a heatmap. The optimal region forms a broad ridge rather than a sharp peak:

\begin{itemize}
  \item The ridge extends approximately 4° in $\Delta\alpha$ and 6° in $\Delta\beta$ around the optimum
  \item Variations of $\pm 2°$ in either angle reduce $|S|$ by less than 0.005
  \item The landscape is smooth and well-behaved, with no additional local maxima detected at the 1° grid resolution employed
\end{itemize}

This robustness indicates that the CHSH correlation structure is not fragile to small misalignments in measurement geometry---a practically important feature for experimental implementations. The near-orthogonal angles predicted numerically align with the Gaussian-approximation analysis of the continuous-angle correlators developed in Sec. 4.8.2, which explains both the ~90° structure and the slight displacement of the optimum.


\begin{figure}[H]
\centering
\includegraphics[width=0.9\textwidth]{figures/fig4_angle_ridge.png}
\caption{Angle-Space Ridge Structure for Optimal CHSH Correlations}
\end{figure}
\normalsize

Figure 4: Angle-Space Ridge Structure for Optimal CHSH Correlations. Heatmap of CHSH amplitude $|S|$ as a function of measurement angle differences ($\Delta\alpha = a' - a$, $\Delta\beta = b' - b$), with a = 0° and b = 45° held fixed. The global optimum (white star) occurs at ($\Delta\alpha^*$, $\Delta\beta^*$) = (95°, 84°), yielding $|S|_{\max} = 2.819 \pm 0.003$. The high-correlation region forms a broad ridge (red/yellow, $|S|$ > 2.7) extending $\approx$10° around the optimum, demonstrating robustness to angular misalignment. The landscape is smooth with no local maxima. Blue regions ($|S|$ < 2.5) correspond to suboptimal measurement geometries. Colorbar shows $|S|$ values from 2.0 to 2.82 (Tsirelson bound marked by dashed line).

\subsubsection{Comparison to Theoretical Predictions}

Standard Bell-CHSH theory predicts optimal angles at:

\[\Delta\alpha = 90°, \quad \Delta\beta = 90° \quad \text{(symmetric case)},\]

or slight modifications depending on the observable model. Our empirically determined optimum (95°, 84°) deviates modestly from this, likely reflecting the specific phase-space structure of Kuramoto coupling.

The asymmetry ($\Delta\alpha$ $\neq$ $\Delta\beta$) suggests that the measurement axes do not align perfectly with the principal directions of the correlation tensor for this system---an issue we return to in the Discussion.

\subsubsection{Summary of Experiment A2}

Key findings from angle optimization:

1. Clean global optimum at (95°, 84°) with $|S|$ = 2.819
2. Broad ridge structure ensuring robustness to angle choice
3. Near-Tsirelson values achievable with continuous-phase measurements

These angles are used for all subsequent experiments to ensure maximum correlation amplitude.


\subsection{Frequency Mismatch Sweet Spot (Experiment A3)}

All previous experiments used frequency mismatch $\Delta\omega = 0.2$, chosen based on exploratory simulations. We now verify that perfect frequency matching ($\Delta\omega = 0$) is not optimal at the ridge point, confirming that moderate detuning enhances correlation amplitude.

\subsubsection{The $\Delta\omega$ Sweep}

We vary $\Delta\omega \in \{0, 0.05, 0.10, 0.20, 0.30\}$ at fixed $K = 0.7$, $\sigma = 0.2$, with measurement angles set to the A2 optimum. For each $\Delta\omega$, we measure $\langle|S|\rangle$ and $r$ over $N = 10$ independent seeds.

Figure 5 (top panel) reveals a clear peak structure:

\[\Delta\omega^* = 0.10 \quad \Rightarrow \quad |S|_{\max} = 2.779 \pm 0.001\]

with degradation on both sides:

\begin{align*}
\Delta\omega = 0.00 &\Rightarrow |S| = 2.766 \pm 0.001 \\
\Delta\omega = 0.20 &\Rightarrow |S| = 2.777 \pm 0.001 \\
\Delta\omega = 0.30 &\Rightarrow |S| = 2.760 \pm 0.001
\end{align*}

\subsubsection{Interpretation: Dynamical Tension vs. Lock Strength}

The observed peak at $\Delta\omega^* = 0.10$ (approximately $0.14K$) demonstrates that perfect frequency matching is suboptimal at this ridge point. The data reveal three distinct regimes:

\textbf{Perfect matching ($\Delta\omega = 0$)}:
\begin{itemize}
  \item Rigid phase locking with minimal phase-space exploration
  \item Lower $|S| = 2.766$, confirming that exact resonance is not optimal
  \item High coherence but reduced correlation amplitude
\end{itemize}

\textbf{Optimal detuning ($\Delta\omega \approx 0.1$)}:
\begin{itemize}
  \item Balanced tension between synchronization and exploration
  \item Maximum $|S| = 2.779$, yielding ~0.5\% enhancement over $\Delta\omega = 0$
  \item Sufficient coupling to maintain coherence with dynamic richness
\end{itemize}

\textbf{Excessive mismatch ($\Delta\omega \gtrsim 0.2$)}:
\begin{itemize}
  \item Frequency difference begins to dominate coupling strength
  \item Systematic decline in $|S|$ (2.777 at $\Delta\omega = 0.2$, 2.760 at $\Delta\omega = 0.3$)
  \item Phase coherence remains robust but CHSH structure weakens
\end{itemize}

At $K = 0.7$ in the ridge regime, the sweet spot occurs near $\Delta\omega^* \approx 0.14K$, suggesting a scaling relationship between optimal detuning and coupling strength. Broader exploration of the $\Delta\omega$--$K$ landscape is left for future work.

\subsubsection{Phase Coherence Remains Robust Across $\Delta\omega$}

Figure 5 (bottom panel) shows that the order parameter $r$ (phase coherence) remains nearly constant across the entire $\Delta\omega$ range tested:

\[r \approx 0.985 \text{ for all } \Delta\omega \in [0, 0.30].\]

This demonstrates that:

\begin{itemize}
  \item Phase-locking persists robustly even as $|S|$ exhibits peaked structure
  \item The CHSH observable is more sensitive to $\Delta\omega$ than the Kuramoto order parameter
  \item Correlation amplitude and phase coherence decouple (cf. Experiment A1)
\end{itemize}


\begin{figure}[H]
\centering
\includegraphics[width=0.9\textwidth]{figures/fig5_delta_omega.png}
\caption{Frequency-Mismatch Sweet Spot and Temporal Memory}
\end{figure}
\normalsize

Figure 5: Frequency Mismatch Sweet Spot at Ridge Point. (Top) Mean CHSH amplitude $|S|$ as a function of detuning $\Delta\omega$ at fixed $K = 0.7$, $\sigma = 0.2$, showing a clear maximum at $\Delta\omega^* = 0.10$ ($|S| = 2.779 \pm 0.001$). Perfect frequency matching ($\Delta\omega = 0$) yields lower correlation amplitude ($|S| = 2.766$), confirming that moderate detuning enhances CHSH structure. Error bars: SEM across $N = 10$ trajectories. (Bottom) Phase coherence $r$ (Kuramoto order parameter) remains nearly constant ($r \approx 0.985$) across the $\Delta\omega$ range, demonstrating robust synchronization independent of detuning. The decoupling between $r$ (flat) and $|S|$ (peaked) reveals that high CHSH correlations require more than simple phase locking---they demand specific dynamical balance. This single-K slice establishes the phenomenon; broader $\Delta\omega$--$K$ landscape exploration is reserved for future work.

\subsubsection{Summary of Experiment A3}

Three principal findings at the ridge point ($K = 0.7$, $\sigma = 0.2$):

\begin{enumerate}
  \item Perfect frequency matching ($\Delta\omega = 0$) is suboptimal: $|S| = 2.766$
  \item Optimal detuning at $\Delta\omega^* = 0.10 \approx 0.14K$ yields $|S| = 2.779$
  \item Phase coherence $r$ remains robust across all $\Delta\omega$, decoupling from CHSH amplitude
\end{enumerate}

This demonstrates that moderate frequency mismatch enhances correlation structure in the ridge regime. Broader exploration of the $\Delta\omega$--$K$ landscape is left for future work.


\subsection{Temporal Coherence Beyond $|S| > 2$ (Experiment B1)}

A central question is whether the return to $|S| < 2$ coincides with the complete loss of temporal structure in the CHSH observable. Experiment B1 demonstrates that autocorrelation of the CHSH time series (measured at lag $\tau = 10$) remains elevated even when instantaneous correlation amplitude has returned to classical values.

\subsubsection{Experimental Design}

We select three noise levels representing qualitatively different regimes:

1. \textbf{Ridge} ($\sigma = 0.2$): Deep in the high-correlation region
2. \textbf{Boundary} ($\sigma = 0.7$): Near the collapse threshold $\sigma_c \approx 0.65$ for $K = 0.7$
3. \textbf{Classical} ($\sigma = 1.0$): Well beyond the classical bound

For each regime, we measure:

\begin{itemize}
  \item CHSH amplitude $\langle|S|\rangle$
  \item Phase coherence $r$
  \item Temporal coherence $\rho_S(\tau = 10)$
\end{itemize}

using $K = 0.7$, $\Delta\omega = 0.2$, and optimal angles.

\subsubsection*{3.4.2\quad Key Findings}

Figure~6 displays the results as a three-panel bar chart. The data show:

\textbf{Ridge regime ($\sigma = 0.2$):}
\[
|S| = 2.774 \pm 0.001, \qquad
r = 0.985 \pm 0.0003, \qquad
\rho_S = 0.762 \pm 0.005.
\]
Strong correlations ($|S| \ge 2$), high synchrony, moderate temporal coherence.

\medskip
\textbf{Boundary regime ($\sigma = 0.7$):}
\[
|S| = 2.228 \pm 0.018, \qquad
r = 0.792 \pm 0.006, \qquad
\rho_S = 0.858 \pm 0.011.
\]
Correlations reduced but still $|S| \ge 2$. Synchrony partially degraded. Temporal coherence increases.

\medskip
\textbf{Classical regime ($\sigma = 1.0$):}
\[
|S| = 1.594 \pm 0.036, \qquad
r = 0.567 \pm 0.013, \qquad
\rho_S = 0.860 \pm 0.005.
\]
Returned to $|S| < 2$. Synchrony weak. \textbf{Temporal coherence remains high.}

\medskip
The persistence of $\rho_S$ at moderate noise, even after $|S|$ drops below 2, is
consistent with the fact that the OU drift rate is largely insensitive to the
distribution's instantaneous width. Section~4.8.4 develops this argument and explains
the mild non-monotonicity near $\sigma \approx 0.7$.


\begin{figure}[H]
\centering
\includegraphics[width=0.9\textwidth]{figures/fig6_memory_panel.png}
\caption{Temporal Memory Panel}
\end{figure}
\normalsize

Figure 6: Memory Persistence Beyond the Classical Bound. Three-panel comparison of dynamical metrics across noise regimes. (Left) CHSH amplitude $|S|$ decreases systematically with noise: Ridge ($\sigma = 0.2$, green) shows $|S| = 2.77$, Boundary ($\sigma = 0.7$, yellow) shows $|S| = 2.23$, Classical ($\sigma = 1.0$, red) shows $|S| = 1.59$. Dashed line marks classical bound $|S| = 2$. (Center) Phase coherence $r$ follows a similar trend but degrades more gradually, indicating partial synchrony persists into the classical regime. (Right) Temporal coherence $\rho_S$ exhibits non-monotonic behavior: it increases from Ridge (0.76) to Boundary (0.86) then remains elevated (0.86) in the Classical regime despite $|S| < 2$. This decoupling demonstrates that temporal memory is maintained through a mechanism distinct from instantaneous correlation amplitude. Error bars: SEM across $N = 15$ trajectories per regime.

\subsubsection{Persistence of Temporal Memory}

The temporal coherence $\rho_S(\tau)$ quantifies the autocorrelation of the CHSH time series $S(t)$ at lag $\tau$. Specifically, $\rho_S(\tau) = \text{Corr}[S(t), S(t+\tau)]$ measures the tendency for the CHSH observable to maintain a consistent value over the timescale $\tau$. A high value of $\rho_S$ indicates that the phase relation between oscillators---and thus the correlation structure encoded in $S(t)$---persists across multiple dynamical periods. This is distinct from the instantaneous correlation amplitude $\langle|S|\rangle$, which reflects the strength of phase alignment at any single moment but says nothing about how long that alignment is maintained. In coupled oscillator systems, temporal memory can persist through slow drift of the relative phase $\Delta\theta(t)$ even when noise has degraded the sharpness of the instantaneous phase distribution. This decoupling arises because $\langle|S|\rangle$ depends on the width of the phase distribution (via diffusion), while $\rho_S$ depends on the drift rate (via coupling-induced restoring forces that resist large phase excursions over time).

The most striking feature of this experiment is that the temporal coherence
$\rho_S(\tau=10)$ remains high ($\approx 0.86$) even when the CHSH amplitude has collapsed to
$|S| = 1.59$, well below the classical bound. At this fixed lag, we observe:

\begin{itemize}
    \item The CHSH time series retains a stable autocorrelation structure.
    \item Values of $S_{\text{inst}}(t)$ continue to correlate with values 10 time units later.
    \item Temporal coherence of the correlation observable persists long after $|S|$ has fallen below 2.
\end{itemize}

This demonstrates a clear separation between \textbf{correlation amplitude} and \textbf{temporal memory}.
A full characterization across multiple lag timescales---via the decay curves in Supplementary Fig. S1---would provide a more complete picture, but the single-lag result already reveals that memory and amplitude respond differently to noise.

This is not trivial. One might expect that noise sufficient to suppress $|S|$ below 2 would also erase temporal structure. Instead, the system shows a \textbf{decoupling}: loss of instantaneous correlation does not imply loss of temporal coherence.

\subsubsection{Non-Monotonic Behavior of $\rho_S$}

A second notable result is the non-monotonic structure of $\rho_S(\tau=10)$.
Temporal coherence reaches its maximum value ($0.858 \pm 0.011$) at the intermediate noise level
$\sigma = 0.7$, exceeding both the ridge ($0.762 \pm 0.005$) and classical ($0.860 \pm 0.005$) regimes.

Several mechanisms may contribute:

\begin{enumerate}
    \item \textbf{Low noise ($\sigma = 0.2$)} --- Rapid, tightly coupled phase dynamics may decorrelate faster at the chosen lag of $\tau = 10$.
    \item \textbf{Intermediate noise ($\sigma = 0.7$)} --- Slower or intermittently locked dynamics can enhance the measured autocorrelation at this particular timescale.
    \item \textbf{High noise ($\sigma = 1.0$)} --- Broader phase diffusion shifts decorrelation times, producing elevated but non-monotonic coherence.
\end{enumerate}

Distinguishing between these mechanisms requires analyzing the full $\rho_S(\tau)$ decay, which we provide in Supplementary Fig. S1. For the present study, we highlight the observation but avoid strong mechanistic claims based on a single fixed lag.

\subsubsection{Implications for Interpretation}

These results establish that temporal coherence and CHSH amplitude reflect \textbf{distinct dynamical structures}:

\begin{itemize}
    \item High $|S|$ does \textbf{not} require perfect memory:
          $\rho_S$ can be modest ($\approx 0.76$) while $|S|$ is large ($2.77$).
    \item Strong memory does \textbf{not} require $|S|>2$:
          $\rho_S$ remains high ($\approx 0.86$) even when $|S|$ has returned to classical values ($1.59$).
    \item Loss of one structure does not immediately imply loss of the other.
\end{itemize}

Thus the transition near $\sigma_c$ separates two processes:
(1) suppression of instantaneous correlations and
(2) degradation of temporal phase memory.
The two unfold on different noise scales, motivating a more refined taxonomy of dynamical regimes (see Section 5).

\subsubsection{Summary of Experiment B1}

Key findings:

\begin{enumerate}
    \item Temporal coherence $\rho_S \approx 0.86$ persists deep into the classical regime ($\sigma = 1.0$).
    \item CHSH correlations ($|S|>2$) and temporal memory represent distinct structures with different noise sensitivities.
    \item The non-monotonic peak in $\rho_S$ near $\sigma = 0.7$ suggests a reorganization of the underlying phase-space dynamics at intermediate noise levels.
\end{enumerate}

This completes the results of Experiment B1.


\section{Discussion}

Several features of the numerical landscape---including the $\sigma_c$(K) scaling, the near-orthogonal measurement geometry, and the detuning-dependent structure---are revisited analytically in Sec. 4.8.

\subsection{What Is Established}

The numerical experiments presented in this work demonstrate that a pair of classically coupled phase oscillators can generate CHSH values $|S|$ > 2 under specific dynamical conditions. These results describe the behavior of this model, within this parameter space, and using a continuous-phase CHSH functional.

Three core findings emerge reliably within the tested regime:

\subsubsection{High CHSH Values in Phase-Coherent Regimes}

When coupling strength is sufficiently large relative to noise and detuning, trajectories remain phase-coherent and yield $|S| > 2$. Maximum values reach:

\[|S|_{\max} = 2.819 \pm 0.003,\]

measured for optimal measurement geometry. These values lie close to the mathematical upper bound $2\sqrt{2}$ for the chosen correlation functional, though no claim is made regarding physical significance of this proximity.\footnote{The value $2\sqrt{2} \approx 2.828$ is the algebraic maximum of the CHSH expression for the specific functional form we employ. In quantum mechanics, this limit is often called the Tsirelson bound after Boris S. Tsirelson (also transliterated as Cirel'son), who derived it in 1980 for entangled qubit pairs with projective measurements. In our classical continuous-phase system, this same algebraic maximum emerges naturally from the geometric structure of the correlation function---not from quantum mechanics or measurement postulates. The coincidence reflects shared mathematical structure (bilinear correlation functionals with trigonometric dependence) rather than physical equivalence. We use ``algebraic maximum'' throughout to avoid importing quantum-specific connotations.}

\subsubsection{Collapse of CHSH Correlations with Increasing Noise}

For each coupling $K$, a noise amplitude $\sigma_c$ can be identified where $|S|$ falls below the classical value 2. Across the interval $K \in [0.3, 0.9]$, $\sigma_c(K)$ is well-approximated by:

\[\sigma_c \approx 0.602K + 0.222 \quad (R^2 = 0.984),\]

reflecting a balance between coupling-driven phase alignment and noise-driven diffusion. This relationship is an empirical trend specific to the tested range; its extrapolation beyond these values is not asserted.

This functional form should be regarded as an empirical fit over the tested interval $K \in [0.3, 0.9]$. Additional measurements at lower and higher coupling strengths may reveal curvature or saturation that is not captured by the present linear approximation.

\subsubsection{A Robust Optimum in Measurement Geometry}

A high-correlation ridge is observed near angle differences $(\Delta\alpha, \Delta\beta) \approx (95°, 84°)$. This geometry maximizes $|S|$ for the continuous-phase CHSH functional used here and remains robust under small perturbations, suggesting a broad region of near-optimal orientations.

\subsubsection{Temporal Coherence Persists Beyond the $|S| > 2$ Regime}

The autocorrelation of the instantaneous CHSH signal, $\rho_S(\tau)$, remains high ($\approx 0.86$) even for $\sigma$ where $|S| \approx 1.6 < 2$. This indicates that the decay of $|S|$ is not synonymous with the loss of temporal structure: phase dynamics can remain coherent on short timescales even after instantaneous correlations fall below the classical threshold.

\subsubsection{Parameter Specificity}

Random parameter configurations rarely yield $|S| > 2$. The effect requires coordinated tuning of coupling, noise, detuning, and measurement geometry. High $|S|$ values therefore arise within a particular dynamical regime, not generically across the model.


\subsection{What Is Not Claimed}

To avoid misinterpretation, we clarify what these results do not imply:

\subsubsection{No Quantum Nonlocality}

All correlations arise from explicit classical coupling $K\sin(\theta_B - \theta_A)$. No entanglement or quantum measurement theory is involved.

\subsubsection{No Violation of Relativistic Causality}

The model contains no spacelike-separated measurements. All influences propagate through the coupling term.

\subsubsection{No Challenge to Bell's Theorem}

Bell's theorem restricts local hidden-variable models with binary outcomes. Our system is not local (due to explicit coupling) and uses continuous correlations; thus $|S|$ > 2 does not contradict Bell in any sense.

\subsubsection{No Detector Loopholes or Post-Selection}

CHSH values are computed from full trajectories with no thresholding or discarded samples.

\subsubsection{No Claim About Quantum Foundations}

We do not assert equivalence between classical oscillators and quantum systems, nor comment on hidden variables or interpretations of quantum mechanics.


\subsection{Interpretation of $|S|>2$ in a Continuous-Phase CHSH Functional}

It is important to clarify what the values $|S|>2$ reported here do, and do
not, imply. In our model the observables are continuous, bounded phase
functionals $X(\phi) = \cos(\Delta\theta + \phi)$, evaluated on a pair of
explicitly coupled classical oscillators. Appendix~A shows that, despite this
continuous structure, the usual classical CHSH bound $|S| \le 2$ still holds
for any local hidden-variable model built from such bounded observables. In
that precise sense, the functional we use is a legitimate CHSH-type quantity
with a well-defined classical ceiling.

However, because the system is entirely classical and the coupling is explicit
in the equations of motion, values $|S|>2$ in this context do not signal
nonlocality or quantum entanglement. Instead, they quantify a strong,
geometry-driven enhancement of correlations produced by the drift--diffusion
dynamics of the phase difference $\Delta\theta(t)$ under tuned choices of
$K$, $\Delta\omega$, $\sigma$, and measurement angles. The ``violation-like''
behavior we observe is therefore best understood as a \emph{classical
correlation amplifier}: the same CHSH functional, with the same classical
bound, applied to a nontrivial continuous phase distribution.

In summary, the significance of $|S|>2$ here is not that a classical system
mimics a quantum nonlocality test, but that a simple two-oscillator model can
be tuned into a regime where its continuous-phase statistics saturate and
exceed the standard CHSH classical limit. This makes the system a useful
benchmark for understanding how structured dynamics in classical fields can
approach, and sometimes numerically resemble, quantum-like correlation
patterns without invoking any nonclassical resources.


\subsection{Mechanistic Insight}

The results can be understood through the geometry and dynamics of the phase difference $\Delta\theta = \theta_B - \theta_A$.

\subsubsection{Effective Potential and Phase Locking}

The coupling term induces an effective potential

\[V(\Delta\theta) = -K\cos(\Delta\theta),\]

which stabilizes $\Delta\theta$ near zero when $K$ dominates detuning and noise. Perfect locking ($\Delta\theta$ = 0) yields trivial correlations, but small fluctuations around the minimum generate structured cosine correlations across measurement angles.

The continuous CHSH functional effectively samples this structure at four rotated projections of $\Delta\theta$. High $|S|$ values arise when:

\begin{itemize}
  \item $\Delta\theta$(t) remains confined near the potential minimum,
  \item but fluctuates enough to avoid a degenerate single-value distribution.
\end{itemize}

\subsubsection{Noise as a Shaping Mechanism}

Noise plays two opposing roles:

\textbf{Low noise:}
Phase locking is too rigid; $\Delta\theta$ explores a narrow region, limiting variation across measurement orientations. $|S|$ is high but not maximized.

\textbf{Moderate noise:}
$\Delta\theta$ explores more of the curved region around the potential minimum, enhancing the contrast between the four CHSH terms. This creates the high-$|S|$ ridge.

\textbf{High noise:}
Noise-induced diffusion overpowers the restoring force; $\Delta\theta$ becomes broad and unstructured, causing $|S|$ to fall below 2.

The collapse threshold $\sigma_c$ marks the point where diffusion time scales become comparable to locking time scales.

\subsubsection{Role of Frequency Mismatch}

A small detuning $\Delta\omega$ introduces persistent rotational drift, preventing $\Delta\theta$ from collapsing into a trivial fixed point. The optimum at $\Delta\omega$ $\approx$ 0.2 arises when detuning provides enough dynamical tension to enrich phase-space exploration without destroying coherence.

Preliminary tests suggest that the optimal detuning lies within a narrow band around $\Delta\omega$ $\approx$ 0.2, though the precise location and width of this optimum may shift as additional low-detuning runs ($\Delta\omega$ = 0 and $\Delta\omega$ = 0.05) are incorporated. These values will help determine whether the enhancement reflects a smooth tension--coherence trade-off or a sharper resonance-like effect.

\subsubsection{Temporal Coherence Beyond the Classical Regime}

Even when $|S|$ < 2, the drift of $\Delta\theta$ can remain slow relative to the sampling timescale, producing high autocorrelation $\rho_S$. Thus:

\begin{itemize}
  \item instantaneous correlation amplitude and
  \item temporal predictability
\end{itemize}

reflect distinct dynamical features. This explains the observed persistence of memory well into the classical regime.


\subsection{Limitations}

Several factors constrain the generality of these results:

\subsubsection{Two-Oscillator Model}

Behavior may differ in larger oscillator networks.

\subsubsection{CHSH Functional Form}

We use a continuous cosine-based CHSH generalization rather than dichotomic $\pm 1$ outcomes. This differs from standard Bell tests and should be interpreted accordingly. Continuous-variable Bell inequalities of this form have precedent in quantum optics, particularly in the Banaszek--Wódkiewicz framework, where phase-space correlators replace dichotomic outcomes and yield analogous CHSH structures.

\subsubsection{Gaussian Additive Noise}

Other noise types may change collapse behavior.

\subsubsection{Numerical Scheme}

Explicit Euler integration may introduce discretization artifacts. Complementary simulations with higher-order integrators would strengthen robustness.

\subsubsection{Parameter Scope}

Results refer to the tested ranges of $K$, $\sigma$, $\Delta\omega$, and measurement angles; behavior outside these ranges may differ.


\subsection{Open Questions and Future Work}

\subsubsection{Structure of $\sigma_c(K)$}

The linear trend $\sigma_c(K) \approx 0.602K + 0.222$ holds robustly over the mid-range of couplings tested here, but the behavior outside this window remains unresolved. The full sweep (including $K < 0.3$ and $K > 1.0$) will determine whether the observed intercept reflects an underlying dynamical feature, a fitting artifact, or local linearity on a globally curved collapse surface.

\subsubsection{Higher-Order Correlations and N > 2}

Extension to larger oscillator networks and multi-party Bell-type inequalities could reveal new collective regimes.

\subsubsection{Other Bell Measures}

Testing CHSH alongside CGLMP, Leggett--Garg, or thresholded binary correlations may clarify the dependence on measurement scheme.

\subsubsection{Non-Gaussian and Colored Noise}

Exploring more realistic noise models may expose different collapse mechanisms.

\subsubsection{Dynamical Modulation}

Time-varying coupling or adaptive feedback could shape correlation structure in ways static coupling does not.

\subsubsection{Phase-Space Topology}

Understanding why the optimal measurement geometry deviates from (90°, 90°) requires a deeper study of the local curvature of $V(\Delta\theta)$, locking basins, and drift symmetries.

\subsubsection{Experimental Implementations}

Candidate physical systems include optomechanical oscillators, phase-locked lasers, Josephson junction arrays, and electrochemical oscillators.


\subsubsection{Broader Context}

The observation of $|S| > 2$ in a classical system with explicit coupling underscores that CHSH values test correlation structure, not "quantumness," when the locality assumption is relaxed. Quantum systems remain unique in producing $|S| > 2$ without explicit interactions, but classical systems with coupling can generate similar numerical values. This distinction illustrates the importance of assumptions---particularly locality and binary outcomes---in interpreting Bell-type inequalities.


\subsubsection{Why This Does Not Trivialize Quantum Correlations}

A natural concern is whether demonstrating $|S| > 2$ in a classical coupled system diminishes the significance of quantum violations of Bell inequalities. We emphasize that it does not. The critical distinction lies in the \textbf{mechanism} by which correlations arise, not merely their numerical magnitude. Quantum systems achieve $|S| > 2$ through entanglement---a form of correlation with no classical analog---and crucially, they do so \textbf{without any physical coupling} between the measured subsystems. In contrast, our classical oscillators are explicitly coupled through the term $K\sin(\theta_B - \theta_A)$ in the equations of motion. This coupling directly transmits information between the oscillators at every timestep, making the system fundamentally non-local in the Bell sense. The high CHSH values we observe are a consequence of this coupling-driven correlation structure, not a violation of locality or a challenge to quantum mechanics. In fact, our results reinforce the foundational distinction: quantum Bell violations are remarkable precisely because they occur \textit{without} such explicit coupling, defying local hidden-variable explanations. Classical coupled systems like ours can mimic the \textit{statistics} but not the \textit{physics}. The two phenomena share mathematical structure (bilinear correlation functionals) but arise from entirely different physical mechanisms. Understanding this difference clarifies rather than obscures the unique role of quantum entanglement in fundamental physics.


\subsubsection{Summary of Discussion}

We have shown that:

1. Within this dynamical model, high CHSH values arise robustly under phase-coherent conditions.
2. A clear correlation collapse emerges as noise increases, with empirical scaling over the tested range.
3. Temporal coherence persists independently of instantaneous CHSH amplitude.
4. The results do not impact quantum nonlocality or Bell's theorem.
5. Mechanistic analysis highlights the roles of the effective potential, noise, and detuning.
6. Several conceptual and experimental avenues remain open.


While the numerical results in Sections 3--4 characterize the CHSH landscape across noise, coupling strength, detuning, and measurement geometry, several of the observed patterns suggest underlying dynamical mechanisms that can be exposed with simple approximations. In particular, the linear mid-range collapse boundary $\sigma_c$(K), the near-orthogonal measurement geometry, the detuning-dependent maximum in $|S|$, and the persistence of temporal memory all point toward coarse drift--diffusion structure in the phase-difference dynamics. To clarify these connections, we provide a minimal analytical framework that captures the leading contributions without attempting a full solution of the nonlinear stochastic equations. This framework is intended to contextualize the numerical trends, not replace them.

\subsection{Analytical Framework}

To complement the numerical results of Sections 3--4, we outline a minimal analytical framework that captures the leading dynamical mechanisms shaping the CHSH landscape. Although the fully nonlinear stochastic system is not analytically tractable, several coarse-grained approximations yield useful insight into (i) the noise--coupling collapse boundary $\sigma_c$(K), (ii) the structure of the optimal measurement geometry, (iii) the emergence of a detuning-dependent maximum in $|S|$, and (iv) the persistence of temporal memory after $|S|$ falls below the classical bound. These approximations are not intended as quantitative predictors across the full parameter range, but rather as organizing principles for understanding the trends visible in Figs. 2--6.

\subsubsection{Phase-Difference Dynamics and Drift--Diffusion Balance}

The system studied numerically is governed by the pair of noisy coupled Kuramoto oscillators

\[\dot{\theta}_A = \omega_A + K\sin(\theta_B - \theta_A) + \sigma\,\xi_A(t),\]

\[\dot{\theta}_B = \omega_B + K\sin(\theta_A - \theta_B) + \sigma\,\xi_B(t),\]

where $\xi_i(t)$ are independent Gaussian white noises with

\[\langle \xi_i(t)\,\xi_j(t')\rangle = \delta_{ij}\,\delta(t-t').\]

Defining the phase difference $\Delta\theta = \theta_B - \theta_A$, subtracting the two equations yields

\[\dot{\Delta\theta} = \Delta\omega - 2K\sin(\Delta\theta) + \sqrt{2}\sigma\,\xi(t),\]

where $\Delta\omega = \omega_B - \omega_A$ and $\xi(t)$ is a new white noise.

\textbf{Linearized dynamics around the locked state}

Near the synchronized state $\Delta\theta \approx 0$,

\[\sin(\Delta\theta) \approx \Delta\theta,\]

and the equation reduces to an Ornstein--Uhlenbeck (OU) process:

\[\dot{\Delta\theta} \approx \Delta\omega - 2K\,\Delta\theta + \sqrt{2}\,\sigma\,\xi(t).\]

For $\Delta\omega = 0$, the stationary variance is the standard OU result

\[\langle (\Delta\theta)^2\rangle_{\mathrm{eq}} = \frac{\sigma^2}{K}.\]

This linear balance between drift ($\propto K$) and diffusion ($\propto \sigma^2$) captures the main dependencies observed in Fig. 2: increasing $K$ narrows the phase distribution and increases $|S|$, whereas increasing $\sigma$ broadens the distribution and suppresses $|S|$.

\textbf{Predicting the collapse boundary} $\boldsymbol{\sigma_c(K)}$

The CHSH amplitude $|S|$ falls below 2 when the width of the phase-difference distribution exceeds a characteristic angular scale $\Delta\theta_c$ set by the measurement geometry. Within the OU approximation,

\[\sqrt{\langle (\Delta\theta)^2\rangle_{\mathrm{eq}}} \approx \Delta\theta_c \quad\Longrightarrow\quad \sigma_c(K) \propto K.\]

The numerical collapse boundary (Fig. 2) indeed exhibits a linear mid-range regime,

\[\sigma_c(K) \approx 0.60\,K + 0.22,\]

consistent with a drift--diffusion balance.

Departures from linearity arise for:
\begin{itemize}
  \item \textbf{low K}, where the restoring force $2K$ is too weak for the linear approximation to hold and the dynamics explore a broad region of the $2\pi$ potential;
  \item \textbf{high K}, where the $\sin(\Delta\theta)$ nonlinearity becomes important and the benefit of increasing $K$ saturates.
\end{itemize}

Both deviations appear clearly in Fig. 2.

\subsubsection{Measurement Geometry and Correlation Structure}

The continuous-angle CHSH functional depends on correlators of the form

\[E(a,b) = \big\langle \cos\!\big[(\theta_A + a) - (\theta_B + b)\big] \big\rangle = \langle \cos(\Delta\theta + \Delta\phi)\rangle,\]

where $\Delta\phi$ = b - a. Cosine-based phase correlations of this type are standard in continuous-variable (CV) Bell-type functionals.

If $\Delta\theta$ is approximately Gaussian with variance $\langle(\Delta\theta)^2\rangle$, then

\[E(a,b) \approx \cos(\Delta\phi)\, \exp\!\left[-\tfrac{1}{2}\langle(\Delta\theta)^2\rangle\right].\]

A CHSH functional constructed from four such correlators is maximized when the four $\Delta\phi$ values sample both regions of positive correlation (near alignment) and negative correlation (near anti-alignment). For cosine functions, this occurs for relative angle separations near 90°.

\textbf{Shift of the optimal geometry}

The numerical optimum in Fig. 4 occurs near (95°, 84°), slightly displaced from the symmetric (90°, 90°). This shift reflects the fact that:
\begin{itemize}
  \item the stationary distribution P($\Delta\theta$) is only approximately Gaussian,
  \item finite noise introduces mild broadening and asymmetric tails, and
  \item small detuning generates slight skewness in the drift.
\end{itemize}

These effects shift the optimal sampling angles away from the perfectly symmetric case while preserving the ~90° structure predicted by the OU-based approximation.

\subsubsection{Detuning-Induced Maximum in $|S|$}

When $\Delta\omega$ $\neq$ 0, the drift term acquires a constant offset:

\[\dot{\Delta\theta} = \Delta\omega - 2K\sin(\Delta\theta) \approx \Delta\omega - 2K\,\Delta\theta \quad (\Delta\theta \text{ small}),\]

which corresponds to a tilted effective potential. This leads to three qualitative regimes:

\begin{itemize}
  \item \textbf{Small detuning ($\Delta\omega \approx 0$)}: The distribution remains narrowly centered around $\Delta\theta = 0$, limiting the range of correlations and producing relatively modest $|S|$.
\end{itemize}

\begin{itemize}
  \item \textbf{Moderate detuning}: Competition between the tilt $\Delta\omega$ and the restoring force $2K$ broadens $P(\Delta\theta)$ enough to enhance the contrast among the correlators $E(a,b)$, increasing $|S|$.
\end{itemize}

\begin{itemize}
  \item \textbf{Large detuning ($\Delta\omega \gg K$)}: The drift toward the shifted equilibrium dominates, the distribution spreads, and $|S|$ decreases.
\end{itemize}

A simple balance condition suggests an intermediate optimum when

\[\Delta\omega \sim cK,\]

with c of order unity. Numerically, c $\approx$ 0.3 yields good agreement with the observed maximum at $\Delta\omega$ $\approx$ 0.2 for K = 0.7 (Fig. 5). This proportionality should be interpreted as an approximate scaling rather than a strict prediction.

\subsubsection{Temporal Memory and the Persistence of $\rho_S$}

Whereas $|S|$ depends on the instantaneous width of the distribution P($\Delta\theta$), the temporal coherence $\rho_S$ reflects correlations in time. In the OU approximation with $\Delta\omega$ = 0,

\[\rho(\tau) = \exp(-2K\tau),\]

indicating that the decay rate is set primarily by the drift strength rather than by the noise level.

\textbf{Decoupling of $|S|$ and $\rho_S$}

Because the drift rate remains roughly constant for moderate noise, the autocorrelation $\rho_S$ can remain high even after $|S|$ falls below 2. This behavior is visible at $\sigma = 1.0$ in Fig. 6, where $|S|$ has fallen to $\approx 1.59$ yet $\rho_S$ remains elevated.

\textbf{Non-monotonicity near $\sigma \approx 0.7$}

The peak in $\rho_S$ around $\sigma \approx 0.7$ likely reflects an interplay between the measurement lag $\tau$, the effective diffusion timescale $\tau_{\text{diff}} \sim 1/\sigma^2$, and the curvature of the potential. The supplemental $\rho_S(\tau)$ curves will provide additional evidence for this interpretation.

In summary, the analytical approximations developed in this section---linearized phase-difference dynamics, OU-type drift--diffusion balance, angle-geometry sensitivity, detuning-induced tension, and the distinct timescales governing temporal memory---provide a consistent theoretical framework for interpreting the numerical results. Although simplified, these models capture the primary mechanisms shaping the CHSH landscape and quantitatively rationalize the empirical trends observed in Figs. 1--5. More detailed analysis, including full nonlinear treatments or Fokker--Planck solutions, lies beyond the scope of the present work but would be a natural direction for future study. We now turn to a brief summary of the main numerical findings and their implications.


\section{Conclusions}

We have shown that a classical system of two coupled phase oscillators---with explicit, dynamical coupling---can robustly produce CHSH values $|S| > 2$. Exceeding the classical CHSH bound is not a fragile outlier effect but a stable feature of a broad region of parameter space, reaching $|S| \approx 2.82$, near the Tsirelson bound. The mechanism is purely classical: deterministic dynamics, local differential equations, and explicit coupling. No quantum entanglement or hidden variables are invoked.

\subsection{Summary of Key Findings}

\textbf{1. Classical dynamics produce strong CHSH correlations}

Across a wide range of couplings $K$ and low noise $\sigma$, the system consistently exhibits $|S| > 2$, with maxima at $|S|_{\max} = 2.819 \pm 0.003$. Classical phase coherence is sufficient to generate correlation patterns traditionally associated with quantum systems.

\textbf{2. A predictable noise-collapse boundary}

Increasing noise reduces correlation strength in a smooth and reproducible way. The collapse threshold follows an empirical linear relation,

\[\sigma_c(K) = 0.602 K + 0.222 \quad (R^2 = 0.984),\]

providing a quantitative boundary between high- and low-correlation regimes. (Full-range sweeps may reveal curvature outside the tested window.)

\textbf{3. Optimal measurement geometry}

Angle scans reveal a broad, smooth ridge with a global optimum at ($\Delta\alpha$, $\Delta\beta$) = (95°, 84°), corresponding to $(a, a', b, b')$ = (0°, 95°, 45°, 129°). The plateau-like geometry ensures robustness to small misalignments---an important feature for experimental implementations.

\textbf{4. Frequency mismatch enhances correlation}

At the ridge point ($K = 0.7$, $\sigma = 0.2$), moderate detuning ($\Delta\omega^* = 0.10 \approx 0.14K$) maximizes $|S|$ by balancing synchronization strength against dynamical tension. Perfect frequency matching ($\Delta\omega = 0$) yields lower correlation amplitude, confirming that exact resonance is suboptimal in this regime.

\textbf{5. Memory persists beyond the classical boundary}

Temporal coherence $\rho_S$ remains high ($\approx$0.86) even after $|S|$ drops below 2. This decoupling between instantaneous correlation amplitude and temporal memory shows that the two are governed by distinct dynamical structures.

\textbf{6. The effect is parameter-specific}

Random parameter choices do not generically yield $|S| > 2$. The optimized configuration sits at the 99th percentile of a 100-sample random ensemble. High-$|S|$ behavior requires deliberate tuning of coupling, mismatch, noise, and measurement geometry.


\subsection{Implications}

\subsubsection{Foundations of Physics}

These results reinforce that $|S| > 2$ signals correlation structure, not quantum ontology. Bell's theorem remains untouched: classical systems with explicit coupling are not local hidden-variable models. Our system simply occupies a different box in the Bell diagram---one where explicit interaction terms generate structured correlations.

\subsubsection{Dynamical Systems and Complex Behavior}

The CHSH functional becomes a new diagnostic tool for classical dynamical systems, complementing phase coherence, Lyapunov spectra, and synchronization indices. It exposes correlation geometry that is invisible to more traditional metrics. This opens the door to CHSH-style analysis of neural oscillators, coupled lasers, and mechanical synchronization networks.

\subsubsection{Toward Experimental Realizations}

Several platforms are well-suited for testing these predictions:

\begin{itemize}
  \item Coupled optomechanical oscillators
  \item Josephson junction arrays
  \item Phase-locked lasers
  \item Electrochemical or biochemical oscillators
\end{itemize}

An experimental demonstration of $|S| > 2$ in a classical setting would validate the framework and enable exploration of noise models and coupling regimes far beyond what simulation permits.


\subsection{Open Questions}

Despite the clarity of the observed phenomena, several issues remain unresolved:

\begin{itemize}
  \item \textbf{Intercept structure}: The nonzero $\sigma_0 \approx 0.22$ may reflect geometric, numerical, or dynamical effects. Full-range sweeps will clarify whether $\sigma_c(K)$ is globally linear or a local tangent to a curved surface.
\end{itemize}

\begin{itemize}
  \item \textbf{Intermediate-noise coherence spike}: The elevated $\rho_S$ at $\sigma = 0.7$ suggests phase-space reorganization, possibly involving slow diffusion or hidden attractors.
\end{itemize}

\begin{itemize}
  \item \textbf{Scaling to larger oscillator networks}: It remains unknown whether N > 2 networks exhibit collective amplification, frustration effects, or new routes to $|S| > 2$.
\end{itemize}

\begin{itemize}
  \item \textbf{Experimental constraints}: Real-world oscillator systems introduce colored noise, time-varying coupling, and finite measurement precision---all avenues for future study.
\end{itemize}


\subsection{Future Directions}

This work forms the first part of a broader research program.

\begin{itemize}
  \item \textbf{Paper 2} will map the full dynamical landscape, including regimes where memory persists without $|S| > 2$. This motivates the Echo-Rich Classical (ECR) taxonomy.
\end{itemize}

\begin{itemize}
  \item Subsequent work will extend to N > 2 oscillators, explore GHZ/Mermin inequalities, analyze non-Gaussian and multiplicative noise, and pursue experimental validation.
\end{itemize}

Longer-term efforts aim to:

\begin{itemize}
  \item Build a geometric theory of $|S| > 2$ in classical phase spaces
  \item Identify universal scaling laws for collapse and memory
  \item Investigate applications to neural dynamics, analog computing, and hybrid classical--quantum architectures
\end{itemize}


\subsection{Closing Remarks}

The central contribution is straightforward:

\textbf{Classical coupled oscillators can produce $|S| > 2$.}

This does not challenge quantum mechanics or undermine Bell's theorem. Instead, it clarifies the conceptual landscape: the CHSH inequality is a statement about correlation geometry, and multiple physical mechanisms can generate such structure.

Classical systems with explicit coupling naturally inhabit a region of that landscape. By charting this territory, we deepen the connection between dynamical systems and correlation theory, revealing structure richer than the classical--quantum dichotomy suggests.


\subsection{Data and Code Availability}

All simulation data, analysis scripts, and figure-generation code are available at:

\begin{verbatim}
https://github.com/AkavartaStudio/RUT-CHSH-Landscape
\end{verbatim}

Reproduction of the full 1,680-trajectory dataset can be performed via:

\begin{verbatim}
bash RUN_ALL_PAPER1.sh
\end{verbatim}

All code is MIT-licensed; all data are CC-BY-4.0.


\subsection{Acknowledgments}

Substantial portions of the conceptual development, mathematical framing, and manuscript preparation were conducted with assistance from large language models, including OpenAI's o1-preview and Anthropic's Claude Sonnet 4.5. These systems were used to refine theoretical arguments, clarify exposition, organize numerical data, and edit text. All experimental design, parameter selection, simulation code, data analysis, and scientific conclusions were reviewed, verified, and approved by the human author, who takes full responsibility for the content of this work.

\subsection{License}

This work is licensed under the Creative Commons Attribution 4.0 International License (CC BY 4.0). You are free to share and adapt this material for any purpose, including commercially, provided you give appropriate credit. Full license: https://creativecommons.org/licenses/by/4.0/


\newpage
\appendix
\section{Appendix A: Classical Bound for Cosine-Based CHSH Functionals}
\setcounter{section}{1}

In this appendix we show that the continuous cosine-based CHSH functional used in this
work satisfies the classical bound $|S| \le 2$ for any classical (local hidden-variable)
probability distribution. Although the observables $X(\phi) = \cos(\Delta\theta + \phi)$
are continuous rather than dichotomic, their boundedness in $[-1,1]$ ensures that the
standard CHSH inequality continues to hold.

\subsection*{A.1 Classical Measurement Model}

Let $\lambda$ denote hidden variables with probability density $\rho(\lambda)$.
Define four classical responses:
\[
A_a(\lambda) = X(a,\lambda), \qquad
A_c(\lambda) = X(c,\lambda),
\]
\[
B_b(\lambda) = X(b,\lambda), \qquad
B_d(\lambda) = X(d,\lambda),
\]
with
\[
-1 \le A_a(\lambda), A_c(\lambda), B_b(\lambda), B_d(\lambda) \le 1.
\]
The CHSH functional is
\[
S = E(a,b) + E(a,d) + E(c,b) - E(c,d),
\]
where each correlation is
\[
E(\alpha,\beta) = \mathbb{E}_\lambda \big[A_\alpha(\lambda) B_\beta(\lambda)\big].
\]

\subsection*{A.2 Deterministic CHSH Expression}

For fixed $\lambda$, define the deterministic CHSH quantity
\[
S(\lambda) = A_a(\lambda)\big[B_b(\lambda) + B_d(\lambda)\big]
             + A_c(\lambda)\big[B_b(\lambda) - B_d(\lambda)\big].
\]
We now show $|S(\lambda)| \le 2$ for all $\lambda$. Taking the expectation over $\lambda$
then yields the desired result.

\subsection*{A.3 Boundedness Argument}

Because all observables lie in $[-1,1]$, two cases exhaust the extremal structure of
$S(\lambda)$:

\paragraph*{Case 1: $B_b(\lambda) = B_d(\lambda)$}

Then
\[
S(\lambda) = 2 A_a(\lambda) B_b(\lambda),
\]
so
\[
|S(\lambda)| \le 2.
\]

\paragraph*{Case 2: $B_b(\lambda) = -B_d(\lambda)$}

Then
\[
S(\lambda) = 2 A_c(\lambda) B_b(\lambda),
\]
so
\[
|S(\lambda)| \le 2.
\]

\paragraph*{General Case.}
For arbitrary $B_b, B_d \in [-1,1]$, the bound follows by convexity.
The extremal values occur when the observables align or anti-align,
both yielding $|S(\lambda)| \le 2$.

\subsection*{A.4 Conclusion}

Since $|S(\lambda)| \le 2$ for all $\lambda$, taking the expectation gives
\[
|S| = \left|\mathbb{E}_\lambda[S(\lambda)]\right| \le \mathbb{E}_\lambda[|S(\lambda)|] \le 2.
\]
This establishes that the cosine-based CHSH functional obeys the classical bound
$|S| \le 2$ for all local hidden-variable models, regardless of whether the
observables are dichotomic or continuous.


\newpage
\section{Appendix B: Analytic Rationale for Deviations from the Canonical CHSH Angles}

The optimum CHSH configuration for dichotomic $\pm 1$ observables occurs at orthogonal measurement differences (90°, 90°). In the present continuous-phase model, however, the maximally correlating angle set is shifted to approximately (84°, 95°). This deviation arises from the fact that the instantaneous phase difference $\Delta\theta(t)$ is not uniformly distributed, but instead exhibits (i) \textbf{non-zero skewness}, (ii) \textbf{finite variance}, and (iii) \textbf{non-Gaussian tails} induced by the drift--diffusion dynamics of the coupled oscillators.

Let
\[
E(a,b) = \langle \cos(\Delta\theta + \phi_{ab})\rangle
\]
where $\phi_{ab}$ is the measurement offset. For a symmetric distribution of $\Delta\theta$ with mean zero, the first derivative of $E(a,b)$ with respect to $\phi$ vanishes at $\phi = 90°$, yielding the familiar sinusoidal CHSH optimum. However, when the distribution acquires small skewness or kurtosis corrections---precisely the case in the ridge regime---the expectation can be expanded perturbatively as
\[
E(a,b) \approx \cos\phi \cdot \langle \cos\Delta\theta\rangle - \sin\phi \cdot \langle \sin\Delta\theta\rangle.
\]

If $\Delta\theta$ were perfectly symmetric, $\langle \sin\Delta\theta\rangle = 0$. But numerical evaluation shows that small asymmetries (typically of order $10^{-2}$--$10^{-1}$) arise from the interplay of detuning $\Delta\omega$, noise $\sigma$, and finite-strength coupling $K$. The CHSH expression
\[
S(\phi_{ab}, \phi_{ad}, \phi_{cb}, \phi_{cd})
\]
then acquires a \textbf{linear response term} proportional to $\langle \sin\Delta\theta\rangle$ which shifts its critical point away from exactly 90°.

To first order in the asymmetry parameter
\[
\epsilon = \langle \sin\Delta\theta\rangle,
\]
one finds that the extremum of $S$ occurs at
\[
\phi^* \approx \frac{\pi}{2} - \frac{\epsilon}{\langle\cos\Delta\theta\rangle}.
\]

For the ridge regime at $K = 0.7$, $\sigma = 0.2$, and $\Delta\omega = 0.2$, the phase difference distribution yields
\[
\epsilon = \langle\sin\Delta\theta\rangle = 0.141, \qquad
\langle\cos\Delta\theta\rangle = 0.975,
\]
extracted from the time window $t \in [5000\tau, 10000\tau]$ after transient decay.

The numerical values $\epsilon = 0.141$ and $\langle\cos\Delta\theta\rangle = 0.975$ appearing in the perturbative correction are measured directly from the ridge regime. The high value of $\langle\cos\Delta\theta\rangle \approx 0.975$ confirms strong phase locking: the oscillators maintain a tightly concentrated phase difference distribution centered near $\Delta\theta \approx 0$. The non-zero skewness parameter $\epsilon = 0.141$ indicates a residual asymmetry in the distribution, reflecting the combined effects of detuning ($\Delta\omega$) and noise-driven fluctuations that bias the phase difference slightly away from perfect symmetry. This asymmetry explains why the optimal CHSH measurement angles shift away from the ideal $0°$, $90°$, and $180°$ settings that would be appropriate for a perfectly symmetric, zero-mean phase difference. Together, these parameters fully determine the perturbed angular configuration.

Substituting these measured values, the extremum of $S$ occurs at
\[
\phi^* \approx \frac{\pi}{2} - \frac{0.141}{0.975} \approx 1.43\,\text{rad} \approx 82°,
\]
in good agreement with the empirically observed optimum of $\approx$84°--86°.

A similar perturbation applies to the complementary angle (nominally 90° in the CHSH construction), producing a small compensating adjustment of opposite sign (yielding $\approx$95°). Together these shifts preserve the relative structure of the CHSH geometry while accommodating the asymmetry of the underlying continuous-phase distribution.

Thus, the departure from the canonical (90°, 90°) configuration is not anomalous; it is a direct, quantitatively predictable consequence of small deviations from symmetry in the $\Delta\theta$ distribution. The perturbed-optimum angles constitute the natural generalization of the CHSH geometry for continuous, correlated, non-Gaussian phase variables.


\newpage
\section{Supplementary Material}

\subsection*{Figure S1: Autocorrelation Decay Curves}

\begin{figure}[H]
\centering
\includegraphics[width=0.9\textwidth]{figures/figS1_autocorr.png}
\caption{Temporal Autocorrelation Decay}
\end{figure}
\normalsize

\textbf{Figure S1: Temporal Autocorrelation Decay.}
Autocorrelation $\rho_S(\tau)$ as a function of lag $\tau$ for three
noise regimes at $K = 0.7$, $\Delta \omega = 0.2$, optimal angles:
(left) $\sigma = 0.2$ (ridge regime), (center) $\sigma = 0.7$
(boundary regime), (right) $\sigma = 1.0$ (classical regime).
At low noise, $\rho_S(\tau)$ decays rapidly ($\tau_{1/2} \approx 15$ steps).
In the transition regime, decay slows significantly. For
$\sigma = 1.0$, memory persists ($\rho_S(100) \approx 0.23$) despite $|S|$
having collapsed to 1.59, demonstrating decoupling of
instantaneous correlation amplitude and temporal coherence.
Shaded regions: SEM across $N = 30$ seeds.

\subsection*{Figure S2: Control Comparison}

\begin{figure}[H]
\centering
\includegraphics[width=0.9\textwidth]{figures/figS1_control_random.png}
\caption{Parameter Specificity Control}
\end{figure}
\normalsize

\textbf{Figure S2: Parameter Specificity Control.}
(A) Histogram of $|S|$ values from $N = 100$ random parameter configurations (gray bars) vs $N = 20$ replications of optimized configuration (red vertical line at $|S| = 2.778$). Random configurations yield mean $|S| = 1.061 \pm 0.659$, with only 7\% exceeding the classical bound (dashed line at $|S| = 2$) and none approaching the Tsirelson bound (dotted line at $2\sqrt{2} \approx 2.828$). The distribution is heavily skewed toward classical values. (B) Cumulative distribution function showing the optimized configuration at the 99th percentile. The very large effect size (Cohen's $d = 3.68$) confirms that high CHSH values are not accidental but require precise parameter tuning. This control validates that the observed $|S| > 2$ values represent a specific dynamical regime, not generic behavior of coupled oscillators.


\end{document}