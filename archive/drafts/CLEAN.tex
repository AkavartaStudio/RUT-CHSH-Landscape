\documentclass[11pt,a4paper]{article}
\usepackage[utf8]{inputenc}
\usepackage{amsmath,amssymb,amsfonts}
\usepackage{graphicx}
\usepackage{hyperref}
\usepackage[margin=1in]{geometry}
\usepackage{mathptmx}
\usepackage{float}
\usepackage{caption}

\title{Continuous-Angle CHSH Correlations in Noisy Coupled Oscillators:\\
A Systematic Parameter-Space Study\\[10pt]
\large Paper 1: The CHSH Landscape}
\author{Kelly McRae}
\date{November 2025}

\hypersetup{
    colorlinks=true,
    linkcolor=blue,
    citecolor=blue,
    urlcolor=blue
}

\begin{document}
\maketitle
\newpage
\tableofcontents
\newpage

\subsection*{Table of Contents}

1. Introduction
   - 1.1 Motivation
   - 1.2 Background and Context
   - 1.3 Related Work
   - 1.4 Objectives of This Paper
   - 1.5 Scope of This Paper
   - 1.6 Contributions
   - 1.7 Outline of the Paper

2. Model and Methods
   - 2.1 Dynamical Model
   - 2.2 Numerical Integration
   - 2.3 Measurement Model: Continuous CHSH
   - 2.4 Parameter Sweeps and Experimental Conditions
   - 2.5 Derived Metrics
   - 2.6 Reproducibility

3. Results
   - 3.1 Noise-Induced Collapse of High CHSH Correlations (Experiment A1)
   - 3.2 Angle Optimization and Ridge Structure (Experiment A2)
   - 3.3 Frequency Mismatch Sweet Spot (Experiment A3)
   - 3.4 Memory Beyond |S| > 2 (Experiment B1)
   - 3.5 Parameter Specificity: Control Comparison (Experiment C1)

4. Discussion
   - 4.1 What is Established
   - 4.2 What is Not Claimed
   - 4.3 Mechanistic Insight
   - 4.4 Limitations
   - 4.5 Open Questions and Future Work
   - 4.6 Broader Context
   - 4.7 Summary of Discussion

5. Conclusions


\section*{Section 1: Introduction}

\subsection*{1. Motivation}

Bell inequalities, and in particular the CHSH form

\[S = E(a,b) - E(a,b') + E(a',b) + E(a',b'),\]

play a central role in distinguishing classical local models from quantum correlations. In quantum theory, exceeding the CHSH bound $|S| \leq 2$ arises from entangled two-qubit states and dichotomic measurements, with a maximum value of $2\sqrt{2}$ (the Tsirelson bound).

Outside of quantum mechanics, many dynamical systems generate non-trivial correlation structures, but their capacity to produce $|S| > 2$ - and the precise conditions under which such high correlations appear - remain poorly understood. Nonlinear phase-coupled oscillators, such as variants of the Kuramoto model, exhibit robust synchronization, persistent phase relations, and rich temporal structure. Yet their CHSH landscape has not been systematically explored.

This work performs a controlled, high-resolution study of CHSH correlations generated by a pair of coupled nonlinear oscillators evolving under noise, detuning, and tunable measurement geometry. The goal is not to model quantum systems, but to characterize the conditions under which a classical continuous dynamical system can produce CHSH values above the classical bound, and to identify the structural features that enable or suppress such high correlations.


\subsection*{2. Background and Context}

Two-oscillator Kuramoto-type models provide a minimal setting where:

\begin{itemize}
  \item phase coherence (synchronization strength)
  \item frequency mismatch (detuning)
  \item coupling strength
  \item external noise
  \item measurement angle geometry
\end{itemize}

jointly determine the instantaneous correlation structure of the phase difference $\Delta\theta$(t).

Earlier exploratory simulations in our laboratory (unpublished) showed that this model can produce CHSH values in the range $2 < |S| < 2.8$. These trials revealed the existence of a high-correlation regime but did not establish its structure or boundaries.

Here we present the first systematic characterization of that regime across noise, detuning, coupling strength, and measurement geometry.


\subsection*{2.1 Related Work}

Bell's theorem and the CHSH inequality were originally formulated in the context of quantum spin and polarization measurements, where dichotomic outcomes and photon counting are natural. Over the last two decades, however, several authors have emphasized that Bell-type correlations can appear in systems that are "classical" in other senses. Spreeuw and others introduced the idea of *classical entanglement* in multimode optical fields, showing that nonseparable correlations between polarization and spatial modes of a single beam can mimic quantum entanglement when described in a Hilbert-space framework [1]. Subsequent work on classical coherence and "entangled" classical light has demonstrated violations of Bell-like inequalities using carefully prepared optical fields and coherence functions, without invoking single-photon states [2,3].

A complementary line of work asks which mathematical ingredients of quantum theory are responsible for Bell violations. De Zela and others have constructed local realist models based on classical optics and inner-product probability measures that reproduce quantum-like correlations, arguing that Hilbert-space geometry, rather than nonlocality per se, underlies Bell's theorem [4]. There are also semiclassical models in which classical fields combined with quantum detection theory can violate Bell inequalities [5]. In a different direction, Gerhardt *et al.* showed that apparent violations can be *faked* with purely classical light by exploiting detector-control loopholes, underscoring the importance of strict experimental assumptions [6].

The present work differs from these approaches in three ways. First, we study a \textbf{deterministic dynamical system}-a pair of locally coupled phase oscillators with tunable coupling strength, frequency mismatch, and additive noise-rather than static optical fields or abstract probabilistic models. Second, we compute the standard \textbf{CHSH functional} directly from the continuous phase trajectories, without photon counting, threshold detection, or post-selection. Third, we map an explicit \textbf{phase diagram} in the space of coupling and noise, identifying a linear collapse boundary ($\sigma_c(K)$), an optimal measurement geometry, and a frequency-mismatch "sweet spot" where $|S|$ nearly saturates the Tsirelson bound. To our knowledge, no previous work has reported $|S| > 2$ emerging from such a noisy deterministic phase-locking mechanism, nor documented the persistence of temporal memory after the CHSH value has returned to the $|S| < 2$ regime.


\subsection*{3. Objectives of This Paper}

The present work aims to answer four foundational questions:

\subsubsection*{(1) Under what conditions do nonlinear phase-coupled oscillators produce CHSH values $|S| > 2$?}

We analyze how these correlations depend on:
\begin{itemize}
  \item coupling strength K,
  \item frequency mismatch $\Delta\omega$,
  \item measurement geometry (angle differences),
  \item and additive noise 