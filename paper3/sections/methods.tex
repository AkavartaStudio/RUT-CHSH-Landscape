% Paper 3 - Methods
% Scaffold - to be written

% PLACEHOLDER

\subsection{Three-oscillator model}
\label{sec:three_osc_model}

We extend the driven two-oscillator system of Papers~1 and~2 to three coupled
oscillators. The equations of motion take the form
\begin{equation}
  \ddot{x}_i + \omega_i^2 x_i = \sum_{j \neq i} K_{ij}(x_j - x_i) + \sigma\,\xi_i(t),
  \label{eq:three_osc}
\end{equation}
where $K_{ij}$ encodes the coupling topology and $\xi_i(t)$ is independent
Gaussian white noise on each oscillator.

% TODO: Define the three topologies (chain, star, triangle)
% TODO: Specify coupling matrices
% TODO: Define CHSH functional for three-oscillator pairs


\subsection{Topology definitions}
\label{sec:topology_defs}

% Chain: 1--2--3
% Star: 2 central, 1 and 3 peripheral
% Triangle: all-to-all

% TODO: Coupling matrices for each


\subsection{CHSH measurement protocol}
\label{sec:chsh_protocol}

% TODO: Describe how CHSH is computed for each oscillator pair
% TODO: Define echo, curvature, and susceptibility for networks


\subsection{Simulation parameters}
\label{sec:sim_params}

% TODO: Grid specifications
% TODO: Integration scheme
% TODO: Ensemble statistics


%=============================================================================
% EXPERIMENTAL OBJECTIVES (Mission 1)
%=============================================================================

\subsection{Experimental objectives}
\label{sec:experimental_objectives}

The primary objective of Mission~1 is to determine how loop curvature $K$
controls the regime structure of a classical triangular oscillator network, and
to identify the critical value $K_c$ at which the frustrated phase loses global
dominance. Specifically, we aim to:

\begin{enumerate}
  \item \textbf{Detect and localize phase transition(s)} between frustrated and
    flat winding states under curvature sweeps;

  \item \textbf{Quantify detuning effects} on regime frequencies, establishing
    whether asymmetry in edge couplings produces systematic changes in
    frustration;

  \item \textbf{Characterize multi-basin behavior} by determining whether
    multiple attractors coexist at fixed $K$, rather than a single phase
    dominating all initial conditions;

  \item \textbf{Separate existence vs.\ selection}, by distinguishing which
    regimes are permitted by the dynamics (existence) from which regimes are
    actually realized under varying initialization manifolds (selection);

  \item \textbf{Establish reproducibility} across initialization families,
    confirming that observed transitions reflect intrinsic loop dynamics rather
    than artifacts of preparation or numerical procedure.
\end{enumerate}

Together, these objectives probe whether classical triangular loops admit
multiple stable regimes under deterministic dynamics, and whether curvature
alone suffices to predict observed outcomes, or whether preparation history must
be treated as a distinct physical parameter.
