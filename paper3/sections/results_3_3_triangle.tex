% Paper 3 - Results Section 3.3: Triangle Topology
% Scaffold - to be written

\subsection{Triangle topology: cyclic memory flow}
\label{sec:triangle_topology}

% PLACEHOLDER

The triangle topology couples all three oscillators symmetrically: each node is
connected to both others. This all-to-all configuration supports cyclic
correlation flow and allows us to study how closed loops affect CHSH memory
structure.

% TODO: CHSH landscape for triangle
% TODO: Symmetry analysis
% TODO: Cyclic vs acyclic comparison
% TODO: Memory threshold analysis

%=============================================================================
% MISSION 1 SUMMARY (TL;DR)
%=============================================================================

\paragraph{Summary of Mission~1 findings.}

Mission~1 establishes the basic regime structure of the triangular loop under
curvature control. Across detuned and symmetric configurations, we confirm the
existence of two dominant classical regimes---frustrated and flat---whose
relative frequencies depend on curvature and preparation. Detuning produces
nontrivial regime statistics, indicating that classical triangles do not settle
into a unique configuration even in nominally simple settings.

A sharp transition occurs at a critical curvature $K_c \approx 0.150$. Below
$K_c$ the system reliably settles into frustrated winding; above $K_c$, flat and
frustrated attractors coexist, revealing clear multi-basin behavior at fixed
$K$. Initialization acts primarily as a selection mechanism, biasing which basin
is realized without altering which regimes are dynamically available. Apparent
``intermittent'' samples in early detuned runs were traced to machine-precision
oscillations near $\phi \approx 0$ and are treated as flat.

These results justify the next stages of investigation: Mission~2 (systematic
initialization manifolds) and Missions~3--4 (loop-charge and braiding behavior),
in order to resolve how history, curvature, and topology jointly determine
regime selection in classical looped oscillators.

%=============================================================================
% MISSION 1 CORE FINDING
%=============================================================================

\paragraph{Result: detuning breaks symmetry and reveals multi-basin structure.}

Detuning of individual edges breaks the na\"ive symmetry and produces distinct
frustrated/flat statistics already before the fine $K$ sweeps, establishing that
classical triangles admit multiple stable regimes generically, not only under
finely tuned curvature.

\paragraph{Result: detuning does not shift the critical curvature.}

Detuning the BC edge, including sign-flipped configurations
($\omega_B = -0.15$, $\omega_C = +0.15$), produces a comparable $\sim$50/50
split near $K_c$ without shifting the critical curvature. This indicates that
detuning affects basin occupation but not regime existence or the location of
$K_c$, reinforcing curvature as the dominant control parameter.

%=============================================================================
% MISSION 1B SUMMARY (drop-in ready)
%=============================================================================

\paragraph{Fine-grained sweeps near the critical coupling.}

In Mission~1B we performed fine-grained detuning and initialization sweeps in
the vicinity of the critical loop-curvature value $K_c$. Across multiple
initialization families (uniform random, symmetric cluster), we observe a sharp
transition at $K_c \approx 0.150$: below $K_c$ the system consistently settles
into a frustrated winding; above $K_c$, both flat and frustrated attractors
coexist. Random initializations sample both basins ($\approx$70/30), whereas
symmetric clustered initializations overwhelmingly collapse into the flat basin,
indicating that while the \emph{existence} of regimes is governed primarily by
$K$, the \emph{selection} of regimes depends on initialization manifold.
``Intermittent'' classifications observed in early detuned runs were determined
to arise from machine-precision oscillations near $\phi \approx 0$ and are
treated as flat in Mission~1B. These results establish a robust phase boundary
and motivate the subsequent exploration of initialization manifolds (Mission~2)
and loop charge / braiding behavior (Missions~3--4).


%=============================================================================
% FIGURES TO PREPARE
%=============================================================================

% Ready to generate from Mission 1/1B data:
%   1. Histogram of frustrated/flat vs detuning
%   2. Critical K sweep curve (transition at K_c ~ 0.150)
%   3. Below vs above K_c regime diagram
%   4. Initialization-manifold comparison plot (random vs symmetric_cluster)

% \begin{figure}[t]
% \centering
% \includegraphics[width=0.9\linewidth]{figs/fig3_triangle_topology}
% \caption{
%     PLACEHOLDER
% }
% \label{fig:fig3_triangle_topology}
% \end{figure}
