% Paper 3 - Introduction
% Scaffold - to be written

% PLACEHOLDER

The two-oscillator system studied in Papers~1 and~2 provides the simplest
setting for CHSH-like correlations in classical dynamics. Paper~1 mapped the
instantaneous CHSH landscape across the $(K,\sigma)$ plane, identifying a
high-$|S|$ ridge and a noise-dependent collapse boundary. Paper~2 resolved the
temporal structure of this landscape, revealing a universal memory threshold
$\sigma_{\mathrm{mem}}\approx 0.002$ that is orders of magnitude below the
instantaneous collapse scale.

These results raise a natural question: how does network topology shape CHSH
memory? In this paper we extend the analysis to three-oscillator systems,
studying three canonical motifs---chain, star, and triangle---that capture
distinct connectivity patterns. Each topology defines a different coupling
graph and therefore a different pathway for correlation flow.

% TODO: Outline the three topologies
% TODO: State the main questions
% TODO: Preview results
