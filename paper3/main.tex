\documentclass[11pt,a4paper]{article}
\usepackage[utf8]{inputenc}
\usepackage{amsmath,amssymb,amsfonts}
\usepackage{graphicx}
\usepackage{hyperref}
\usepackage{cleveref}
\usepackage[margin=1in]{geometry}
\usepackage{mathptmx}
\usepackage{float}
\usepackage[font=small,labelfont=bf]{caption}
\usepackage{parskip}
\setlength{\parindent}{15pt}
\setlength{\parskip}{0pt}
\usepackage{physics}
\usepackage{titlesec}
\usepackage{tocloft}

% Equation spacing
\setlength{\abovedisplayskip}{10pt}
\setlength{\belowdisplayskip}{10pt}

% Section header formatting
\titleformat{\section}{\normalsize\normalfont\bfseries\Large}{\thesection}{1em}{}[\normalsize]
\titleformat{\subsection}{\normalsize\normalfont\bfseries\large}{\thesubsection}{1em}{}[\normalsize]
\titleformat{\subsubsection}{\normalsize\normalfont\bfseries}{\thesubsubsection}{1em}{}[\normalsize]

% TOC formatting
\cftsetpnumwidth{2em}
\cftsetrmarg{3em}

\title{CHSH Memory in Small Oscillator Networks:\\
Topology, Interference, and Collective Echo\\[10pt]
\large Paper 3: Network Motifs}
\author{Kelly McRae}
\date{December 2025}

\hypersetup{
    colorlinks=true,
    linkcolor=blue,
    citecolor=blue,
    urlcolor=blue
}

\begin{document}
\normalfont
\maketitle

\begin{abstract}
% PLACEHOLDER - To be written
Papers~1 and~2 established the CHSH landscape and its temporal structure for a
driven two-oscillator system. Here we extend the analysis to three-oscillator
networks, examining how topology shapes CHSH memory. We study three canonical
motifs---chain, star, and triangle---and characterize their echo surfaces,
curvature profiles, and susceptibility maps. The chain topology exhibits
end-to-end memory decay with a characteristic length scale. The star motif
shows hub-dominated correlations with spoke-to-spoke interference. The triangle
configuration displays cyclic memory flow and topology-dependent phase
structure. Across all motifs, the universal memory threshold
$\sigma_{\mathrm{mem}}\approx 0.002$ persists, confirming its independence from
network structure. These results complete the trilogy by demonstrating that
CHSH memory is a topological observable whose geometry reflects the underlying
coupling graph.
\end{abstract}

\newpage
\tableofcontents
\newpage


%=============================================================================
\section{Introduction}
%=============================================================================

% Paper 3 - Introduction
% Scaffold - to be written

% PLACEHOLDER

The two-oscillator system studied in Papers~1 and~2 provides the simplest
setting for CHSH-like correlations in classical dynamics. Paper~1 mapped the
instantaneous CHSH landscape across the $(K,\sigma)$ plane, identifying a
high-$|S|$ ridge and a noise-dependent collapse boundary. Paper~2 resolved the
temporal structure of this landscape, revealing a universal memory threshold
$\sigma_{\mathrm{mem}}\approx 0.002$ that is orders of magnitude below the
instantaneous collapse scale.

These results raise a natural question: how does network topology shape CHSH
memory? In this paper we extend the analysis to three-oscillator systems,
studying three canonical motifs---chain, star, and triangle---that capture
distinct connectivity patterns. Each topology defines a different coupling
graph and therefore a different pathway for correlation flow.

% TODO: Outline the three topologies
% TODO: State the main questions
% TODO: Preview results



%=============================================================================
\section{Methods}
%=============================================================================

% Paper 3 - Methods
% Scaffold - to be written

% PLACEHOLDER

\subsection{Three-oscillator model}
\label{sec:three_osc_model}

We extend the driven two-oscillator system of Papers~1 and~2 to three coupled
oscillators. The equations of motion take the form
\begin{equation}
  \ddot{x}_i + \omega_i^2 x_i = \sum_{j \neq i} K_{ij}(x_j - x_i) + \sigma\,\xi_i(t),
  \label{eq:three_osc}
\end{equation}
where $K_{ij}$ encodes the coupling topology and $\xi_i(t)$ is independent
Gaussian white noise on each oscillator.

% TODO: Define the three topologies (chain, star, triangle)
% TODO: Specify coupling matrices
% TODO: Define CHSH functional for three-oscillator pairs


\subsection{Topology definitions}
\label{sec:topology_defs}

% Chain: 1--2--3
% Star: 2 central, 1 and 3 peripheral
% Triangle: all-to-all

% TODO: Coupling matrices for each


\subsection{CHSH measurement protocol}
\label{sec:chsh_protocol}

% TODO: Describe how CHSH is computed for each oscillator pair
% TODO: Define echo, curvature, and susceptibility for networks


\subsection{Simulation parameters}
\label{sec:sim_params}

% TODO: Grid specifications
% TODO: Integration scheme
% TODO: Ensemble statistics


%=============================================================================
% EXPERIMENTAL OBJECTIVES (Mission 1)
%=============================================================================

\subsection{Experimental objectives}
\label{sec:experimental_objectives}

The primary objective of Mission~1 is to determine how loop curvature $K$
controls the regime structure of a classical triangular oscillator network, and
to identify the critical value $K_c$ at which the frustrated phase loses global
dominance. Specifically, we aim to:

\begin{enumerate}
  \item \textbf{Detect and localize phase transition(s)} between frustrated and
    flat winding states under curvature sweeps;

  \item \textbf{Quantify detuning effects} on regime frequencies, establishing
    whether asymmetry in edge couplings produces systematic changes in
    frustration;

  \item \textbf{Characterize multi-basin behavior} by determining whether
    multiple attractors coexist at fixed $K$, rather than a single phase
    dominating all initial conditions;

  \item \textbf{Separate existence vs.\ selection}, by distinguishing which
    regimes are permitted by the dynamics (existence) from which regimes are
    actually realized under varying initialization manifolds (selection);

  \item \textbf{Establish reproducibility} across initialization families,
    confirming that observed transitions reflect intrinsic loop dynamics rather
    than artifacts of preparation or numerical procedure.
\end{enumerate}

Together, these objectives probe whether classical triangular loops admit
multiple stable regimes under deterministic dynamics, and whether curvature
alone suffices to predict observed outcomes, or whether preparation history must
be treated as a distinct physical parameter.



%=============================================================================
\section{Results}
%=============================================================================

% Section 3.1 - Chain Topology
% Paper 3 - Results Section 3.1: Chain Topology
% Scaffold - to be written

\subsection{Chain topology: end-to-end memory decay}
\label{sec:chain_topology}

% PLACEHOLDER

The chain topology (1--2--3) represents the simplest extension of the
two-oscillator system: oscillators 1 and 3 are coupled only through the
intermediate node 2. This configuration allows us to probe how CHSH memory
propagates along a linear pathway and whether end-to-end correlations (between
oscillators 1 and 3) differ from nearest-neighbor correlations (1--2 or 2--3).

% TODO: CHSH landscape for chain
% TODO: Echo surfaces for each pair
% TODO: End-to-end vs nearest-neighbor comparison
% TODO: Memory threshold analysis


%=============================================================================
% FIGURE
%=============================================================================

% \begin{figure}[t]
% \centering
% \includegraphics[width=0.9\linewidth]{figs/fig1_chain_topology}
% \caption{
%     PLACEHOLDER
% }
% \label{fig:fig1_chain_topology}
% \end{figure}



% Section 3.2 - Star Topology
% Paper 3 - Results Section 3.2: Star Topology
% Scaffold - to be written

\subsection{Star topology: hub-dominated correlations}
\label{sec:star_topology}

% PLACEHOLDER

The star topology places oscillator 2 at the center (hub), with oscillators 1
and 3 as peripheral nodes (spokes). Each spoke is coupled to the hub but not to
each other. This configuration allows us to study hub-dominated dynamics and
whether spoke-to-spoke correlations emerge through interference at the central
node.

% TODO: CHSH landscape for star
% TODO: Hub-spoke vs spoke-spoke comparison
% TODO: Interference effects
% TODO: Memory threshold analysis


%=============================================================================
% FIGURE
%=============================================================================

% \begin{figure}[t]
% \centering
% \includegraphics[width=0.9\linewidth]{figs/fig2_star_topology}
% \caption{
%     PLACEHOLDER
% }
% \label{fig:fig2_star_topology}
% \end{figure}



% Section 3.3 - Triangle Topology
% Paper 3 - Results Section 3.3: Triangle Topology
% Scaffold - to be written

\subsection{Triangle topology: cyclic memory flow}
\label{sec:triangle_topology}

% PLACEHOLDER

The triangle topology couples all three oscillators symmetrically: each node is
connected to both others. This all-to-all configuration supports cyclic
correlation flow and allows us to study how closed loops affect CHSH memory
structure.

% TODO: CHSH landscape for triangle
% TODO: Symmetry analysis
% TODO: Cyclic vs acyclic comparison
% TODO: Memory threshold analysis

%=============================================================================
% MISSION 1 SUMMARY (TL;DR)
%=============================================================================

\paragraph{Summary of Mission~1 findings.}

Mission~1 establishes the basic regime structure of the triangular loop under
curvature control. Across detuned and symmetric configurations, we confirm the
existence of two dominant classical regimes---frustrated and flat---whose
relative frequencies depend on curvature and preparation. Detuning produces
nontrivial regime statistics, indicating that classical triangles do not settle
into a unique configuration even in nominally simple settings.

A sharp transition occurs at a critical curvature $K_c \approx 0.150$. Below
$K_c$ the system reliably settles into frustrated winding; above $K_c$, flat and
frustrated attractors coexist, revealing clear multi-basin behavior at fixed
$K$. Initialization acts primarily as a selection mechanism, biasing which basin
is realized without altering which regimes are dynamically available. Apparent
``intermittent'' samples in early detuned runs were traced to machine-precision
oscillations near $\phi \approx 0$ and are treated as flat.

These results justify the next stages of investigation: Mission~2 (systematic
initialization manifolds) and Missions~3--4 (loop-charge and braiding behavior),
in order to resolve how history, curvature, and topology jointly determine
regime selection in classical looped oscillators.

%=============================================================================
% MISSION 1 CORE FINDING
%=============================================================================

\paragraph{Result: detuning breaks symmetry and reveals multi-basin structure.}

Detuning of individual edges breaks the na\"ive symmetry and produces distinct
frustrated/flat statistics already before the fine $K$ sweeps, establishing that
classical triangles admit multiple stable regimes generically, not only under
finely tuned curvature.

\paragraph{Result: detuning does not shift the critical curvature.}

Detuning the BC edge, including sign-flipped configurations
($\omega_B = -0.15$, $\omega_C = +0.15$), produces a comparable $\sim$50/50
split near $K_c$ without shifting the critical curvature. This indicates that
detuning affects basin occupation but not regime existence or the location of
$K_c$, reinforcing curvature as the dominant control parameter.

%=============================================================================
% MISSION 1B SUMMARY (drop-in ready)
%=============================================================================

\paragraph{Fine-grained sweeps near the critical coupling.}

In Mission~1B we performed fine-grained detuning and initialization sweeps in
the vicinity of the critical loop-curvature value $K_c$. Across multiple
initialization families (uniform random, symmetric cluster), we observe a sharp
transition at $K_c \approx 0.150$: below $K_c$ the system consistently settles
into a frustrated winding; above $K_c$, both flat and frustrated attractors
coexist. Random initializations sample both basins ($\approx$70/30), whereas
symmetric clustered initializations overwhelmingly collapse into the flat basin,
indicating that while the \emph{existence} of regimes is governed primarily by
$K$, the \emph{selection} of regimes depends on initialization manifold.
``Intermittent'' classifications observed in early detuned runs were determined
to arise from machine-precision oscillations near $\phi \approx 0$ and are
treated as flat in Mission~1B. These results establish a robust phase boundary
and motivate the subsequent exploration of initialization manifolds (Mission~2)
and loop charge / braiding behavior (Missions~3--4).


%=============================================================================
% FIGURES TO PREPARE
%=============================================================================

% Ready to generate from Mission 1/1B data:
%   1. Histogram of frustrated/flat vs detuning
%   2. Critical K sweep curve (transition at K_c ~ 0.150)
%   3. Below vs above K_c regime diagram
%   4. Initialization-manifold comparison plot (random vs symmetric_cluster)

% \begin{figure}[t]
% \centering
% \includegraphics[width=0.9\linewidth]{figs/fig3_triangle_topology}
% \caption{
%     PLACEHOLDER
% }
% \label{fig:fig3_triangle_topology}
% \end{figure}



% Section 3.4 - Cross-Topology Comparison
% Paper 3 - Results Section 3.4: Cross-Topology Comparison
% Scaffold - to be written

\subsection{Cross-topology comparison}
\label{sec:cross_topology}

% PLACEHOLDER

Having characterized each topology individually, we now compare their CHSH
memory structures directly. This comparison reveals which features are
universal (appearing in all topologies) and which are topology-specific.

% TODO: Memory threshold comparison across topologies
% TODO: Echo surface comparison
% TODO: Curvature and susceptibility comparison
% TODO: Universal vs topology-specific features


%=============================================================================
% FIGURE
%=============================================================================

% \begin{figure}[t]
% \centering
% \includegraphics[width=0.9\linewidth]{figs/fig4_topology_comparison}
% \caption{
%     PLACEHOLDER
% }
% \label{fig:fig4_topology_comparison}
% \end{figure}



% Section 3.5 - Synthesis
% Paper 3 - Results Section 3.5: Synthesis
% Scaffold - to be written

\subsection{Synthesis: topology as a CHSH observable}
\label{sec:topology_synthesis}

% PLACEHOLDER

The results above establish that network topology shapes CHSH memory in
specific, predictable ways. Here we synthesize these findings into a unified
picture of how coupling graphs determine correlation structure.

% TODO: Summary of topology-dependent effects
% TODO: Universal memory threshold across all topologies
% TODO: Implications for larger networks
% TODO: Connection to Papers 1 and 2



%=============================================================================
\section{Discussion}
%=============================================================================

% Paper 3 - Discussion
% Scaffold - to be written

% PLACEHOLDER

The extension from two-oscillator to three-oscillator systems reveals that CHSH
memory is fundamentally shaped by network topology. Each motif---chain, star,
and triangle---exhibits distinct correlation patterns while sharing the
universal memory threshold $\sigma_{\mathrm{mem}}\approx 0.002$ established in
Paper~2.

% TODO: Interpret topology-dependent effects
% TODO: Connect to broader network theory
% TODO: Implications for larger systems
% TODO: Limitations and future directions



%=============================================================================
\section{Conclusion}
%=============================================================================

% Paper 2 - Conclusion
% Camera-ready draft from Kelly

This work extends the CHSH landscape introduced in Paper~1 by resolving its
temporal structure across coupling and noise. By measuring lag-resolved
correlations, curvature, echo strength, and susceptibility across a
high-resolution $(K,\sigma)$ grid, we find that CHSH memory in classical
two--oscillator dynamics is organized into three robust regimes: a
noise-dominated basin of trivial memory, a structured intermediate band with
mid-range enhancement, and a narrow ridge of strong instantaneous coherence but
vanishing temporal persistence.

The discovery of a universal memory threshold
$\sigma_{\mathrm{mem}} \approx 0.002$, far below the CHSH collapse curve
$\sigma_{\mathrm{c}}(K)$, highlights a sharp separation of scales between
instantaneous and lagged correlations. This separation imposes a concrete
limitation on the temporal coherence of CHSH-like dynamics: even systems that
exhibit strong instantaneous geometry cannot retain long-lag memory under
noise. The curvature and susceptibility analyses further show that the
intermediate regime contains nontrivial temporal structure that cannot be
inferred from the CHSH functional alone.

These findings suggest a new perspective for interpreting CHSH-like dynamics:
rather than a single coherent structure, the CHSH ridge is embedded within a
multi-layered temporal environment whose properties depend sensitively on both
lag and noise. This layered organization provides a principled vocabulary for
describing ``instantaneous coherence'' versus ``temporal coherence'' in
classical systems and sets the stage for extending the framework to larger
networks.

Paper~3 applies this temporal vocabulary to three-oscillator motifs---chain,
star, and triangle topologies---revealing how CHSH-like memory structures behave
when embedded in small networks with frustration, mediated coupling, and
cyclic flows. Together, the two papers establish a coherent path from
two-oscillator CHSH geometry to the topographic analysis of observer fields in
larger classical systems.



%=============================================================================
% References
%=============================================================================

\bibliographystyle{unsrt}
\bibliography{bib/references}


%=============================================================================
% Appendices (if needed)
%=============================================================================

% \appendix
% \input{sections/appendix}


\end{document}
